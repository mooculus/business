\documentclass{ximera}

\input{../preamble.tex}


\title[Break-Ground:]{We can figure it out}

\begin{document}
\begin{abstract}
Two young mathematicians discuss the derivative of inverse functions.
\end{abstract}
\maketitle

Check out this dialogue between two calculus students (based on a true
story):

\begin{dialogue}
\item[Devyn] Riley, I have a calculus question.
\item[Riley] Hit me with it.
\item[Devyn] What's the derivative of $\arctan(x)$?
\item[Riley] Hmmm\dots we haven't talked about that yet in our class.
\item[Devyn] I know! But maybe we can figure it out.
\item[Riley] Well, if

\[ 
f(x)=e^{x} \text{ then } f^{-1}(x)=\ln (x) 
\]
so
\[
\ln(x) =  e^{-x} 
\]
  and now we can use the chain rule to take its derivative
  \begin{align*}
    \ddx \ln(x) &= \ddx e^{-x} \\
    &= -e^{-x}\\
    &= -\ln(x)\\
   \end{align*}
\item[Devyn] But is this right?
\end{dialogue}

Let's see if we can figure out if Devyn and Riley are correct. Start by looking at a plot of $y = \ln(x)$:

\begin{image}
\begin{tikzpicture}
	\begin{axis}[
            xmin=-6,xmax=6,ymin=-2,ymax=2,
            axis lines=center,
            ytick={0, -1.57,1.57},
            width=9in,
            height=2.5in,
            yticklabels={$0$, $-2$,$2$},
            xtick={0},
            unit vector ratio*=1 1 1,
            xlabel=$x$, ylabel=$y$,
            every axis y label/.style={at=(current axis.above origin),anchor=south},
            every axis x label/.style={at=(current axis.right of origin),anchor=west},
          ]        
          \addplot [very thick, penColor3!20!penColor2, samples=100,smooth, domain=(-6:6)] {ln(x)};
          \addplot [textColor,dashed] plot coordinates {(-6,-1.57) (6,-1.57)};
          \addplot [textColor,dashed] plot coordinates {(-6,1.57) (6,1.57)};
        \end{axis}
\end{tikzpicture}
%% \caption{Here we see a plot of $\arctan(y)$, the inverse function of
%% $\tan(\theta)$ when it is restricted to the interval $(-\pi/2,\pi/2)$.}
\end{image}

\begin{problem}
  Let $f(x) = ln(x)$. Use the plot above to determine the sign of the derivative of $f$.  
  \[
   f'(x)
  \begin{prompt}
     \answer{ >}
  \end{prompt} 0
  \]
\end{problem}

On the other hand,

\begin{problem}
  What is the sign of $-e^{-x}$?
  \[
  -e^{-x} 
  \begin{prompt}
     \answer{ <}
  \end{prompt} 0
  \]
\end{problem}

\begin{problem}
In light of the problems above, is it possible that
\[
\ddx \ln(x) = -e^{-x}?
\]
\begin{multipleChoice}
	\choice{yes}
	\choice[correct]{no}
\end{multipleChoice}
\end{problem}

\begin{problem}
	When our friends wrote $\ln(x) =f^{-1}(x)$ wheren $f(x)=e^{x}$, what do they think the ``$-1$'' represents?  Are they correct?
	\begin{freeResponse}
		Riley thinks that we can use the power rule on the $-1$, which tells us that the students are using $-1$ as an exponent for the exponential function.  However, in the case of inverse functions such as $\ln(x)$, the $-1$ is not an exponent.
	\end{freeResponse}
\end{problem}


% This gets at what the notation sin^{-1} x means, and what the inverse function theorem is saying

% Contrast it with a geometric interpretation that since the inverse is the fxn flipped over the line y=x, we should have this info once we know f'

% becaue we want to actually have two answers that both seem reasonable and demand that they student resolve the contradiction in mathematics

%%\input{../leveledQuestions.tex}

\end{document}
