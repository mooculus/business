\documentclass{ximera}

%\usepackage{todonotes}

\newcommand{\todo}{}

%\usepackage{esint} % for \oiint
\graphicspath{
{./}
{functionsOfSeveralVariables/}
{normalVectors/}
{lagrangeMultipliers/}
{vectorFields/}
{greensTheorem/}
{shapeOfThingsToCome/}
}


\usepackage{tkz-euclide}

\tikzset{>=stealth} %% cool arrow head
\tikzset{shorten <>/.style={ shorten >=#1, shorten <=#1 } } %% allows shorter vectors

\usetikzlibrary{backgrounds} %% for boxes around graphs
\usetikzlibrary{shapes,positioning}  %% Clouds and stars
\usetikzlibrary{matrix} %% for matrix
\usetikzlibrary{patterns} 
\usepgfplotslibrary{polar} %% for polar plots

\usepackage[makeroom]{cancel} %% for strike outs
%\usepackage{mathtools} %% for pretty underbrace % Breaks Ximera
\usepackage{multicol}
\usepackage{pgffor} %% required for integral for loops


%% http://tex.stackexchange.com/questions/66490/drawing-a-tikz-arc-specifying-the-center
%% Draws beach ball
\tikzset{pics/carc/.style args={#1:#2:#3}{code={\draw[pic actions] (#1:#3) arc(#1:#2:#3);}}}



\usepackage{array}
\setlength{\extrarowheight}{+.1cm}   
\newdimen\digitwidth
\settowidth\digitwidth{9}
\def\divrule#1#2{
\noalign{\moveright#1\digitwidth
\vbox{\hrule width#2\digitwidth}}}





\newcommand{\RR}{\mathbb R}
\newcommand{\R}{\mathbb R}
\newcommand{\N}{\mathbb N}
\newcommand{\Z}{\mathbb Z}

%\newcommand{\sage}{\textsf{SageMath}}


%\renewcommand{\d}{\,d\!}
\renewcommand{\d}{\mathop{}\!d}
\newcommand{\dd}[2][]{\frac{\d #1}{\d #2}}
\newcommand{\pp}[2][]{\frac{\partial #1}{\partial #2}}
\renewcommand{\l}{\ell}
\newcommand{\ddx}{\frac{d}{\d x}}

\newcommand{\zeroOverZero}{\ensuremath{\boldsymbol{\tfrac{0}{0}}}}
\newcommand{\inftyOverInfty}{\ensuremath{\boldsymbol{\tfrac{\infty}{\infty}}}}
\newcommand{\zeroOverInfty}{\ensuremath{\boldsymbol{\tfrac{0}{\infty}}}}
\newcommand{\zeroTimesInfty}{\ensuremath{\small\boldsymbol{0\cdot \infty}}}
\newcommand{\inftyMinusInfty}{\ensuremath{\small\boldsymbol{\infty - \infty}}}
\newcommand{\oneToInfty}{\ensuremath{\boldsymbol{1^\infty}}}
\newcommand{\zeroToZero}{\ensuremath{\boldsymbol{0^0}}}
\newcommand{\inftyToZero}{\ensuremath{\boldsymbol{\infty^0}}}



\newcommand{\numOverZero}{\ensuremath{\boldsymbol{\tfrac{\#}{0}}}}
\newcommand{\dfn}{\textbf}
%\newcommand{\unit}{\,\mathrm}
\newcommand{\unit}{\mathop{}\!\mathrm}
\newcommand{\eval}[1]{\bigg[ #1 \bigg]}
\newcommand{\seq}[1]{\left( #1 \right)}
\renewcommand{\epsilon}{\varepsilon}
\renewcommand{\phi}{\varphi}


\renewcommand{\iff}{\Leftrightarrow}

\DeclareMathOperator{\arccot}{arccot}
\DeclareMathOperator{\arcsec}{arcsec}
\DeclareMathOperator{\arccsc}{arccsc}
\DeclareMathOperator{\si}{Si}
\DeclareMathOperator{\proj}{\vec{proj}}
\DeclareMathOperator{\scal}{scal}
\DeclareMathOperator{\sign}{sign}


%% \newcommand{\tightoverset}[2]{% for arrow vec
%%   \mathop{#2}\limits^{\vbox to -.5ex{\kern-0.75ex\hbox{$#1$}\vss}}}
\newcommand{\arrowvec}{\overrightarrow}
%\renewcommand{\vec}[1]{\arrowvec{\mathbf{#1}}}
\renewcommand{\vec}{\mathbf}
\newcommand{\veci}{{\boldsymbol{\hat{\imath}}}}
\newcommand{\vecj}{{\boldsymbol{\hat{\jmath}}}}
\newcommand{\veck}{{\boldsymbol{\hat{k}}}}
\newcommand{\vecl}{\boldsymbol{\l}}
\newcommand{\uvec}[1]{\mathbf{\hat{#1}}}
\newcommand{\utan}{\mathbf{\hat{t}}}
\newcommand{\unormal}{\mathbf{\hat{n}}}
\newcommand{\ubinormal}{\mathbf{\hat{b}}}

\newcommand{\dotp}{\bullet}
\newcommand{\cross}{\boldsymbol\times}
\newcommand{\grad}{\boldsymbol\nabla}
\newcommand{\divergence}{\grad\dotp}
\newcommand{\curl}{\grad\cross}
%\DeclareMathOperator{\divergence}{divergence}
%\DeclareMathOperator{\curl}[1]{\grad\cross #1}
\newcommand{\lto}{\mathop{\longrightarrow\,}\limits}

\renewcommand{\bar}{\overline}

\colorlet{textColor}{black} 
\colorlet{background}{white}
\colorlet{penColor}{blue!50!black} % Color of a curve in a plot
\colorlet{penColor2}{red!50!black}% Color of a curve in a plot
\colorlet{penColor3}{red!50!blue} % Color of a curve in a plot
\colorlet{penColor4}{green!50!black} % Color of a curve in a plot
\colorlet{penColor5}{orange!80!black} % Color of a curve in a plot
\colorlet{penColor6}{yellow!70!black} % Color of a curve in a plot
\colorlet{fill1}{penColor!20} % Color of fill in a plot
\colorlet{fill2}{penColor2!20} % Color of fill in a plot
\colorlet{fillp}{fill1} % Color of positive area
\colorlet{filln}{penColor2!20} % Color of negative area
\colorlet{fill3}{penColor3!20} % Fill
\colorlet{fill4}{penColor4!20} % Fill
\colorlet{fill5}{penColor5!20} % Fill
\colorlet{gridColor}{gray!50} % Color of grid in a plot

\newcommand{\surfaceColor}{violet}
\newcommand{\surfaceColorTwo}{redyellow}
\newcommand{\sliceColor}{greenyellow}




\pgfmathdeclarefunction{gauss}{2}{% gives gaussian
  \pgfmathparse{1/(#2*sqrt(2*pi))*exp(-((x-#1)^2)/(2*#2^2))}%
}


%%%%%%%%%%%%%
%% Vectors
%%%%%%%%%%%%%

%% Simple horiz vectors
\renewcommand{\vector}[1]{\left\langle #1\right\rangle}


%% %% Complex Horiz Vectors with angle brackets
%% \makeatletter
%% \renewcommand{\vector}[2][ , ]{\left\langle%
%%   \def\nextitem{\def\nextitem{#1}}%
%%   \@for \el:=#2\do{\nextitem\el}\right\rangle%
%% }
%% \makeatother

%% %% Vertical Vectors
%% \def\vector#1{\begin{bmatrix}\vecListA#1,,\end{bmatrix}}
%% \def\vecListA#1,{\if,#1,\else #1\cr \expandafter \vecListA \fi}

%%%%%%%%%%%%%
%% End of vectors
%%%%%%%%%%%%%

%\newcommand{\fullwidth}{}
%\newcommand{\normalwidth}{}



%% makes a snazzy t-chart for evaluating functions
%\newenvironment{tchart}{\rowcolors{2}{}{background!90!textColor}\array}{\endarray}

%%This is to help with formatting on future title pages.
\newenvironment{sectionOutcomes}{}{} 



%% Flowchart stuff
%\tikzstyle{startstop} = [rectangle, rounded corners, minimum width=3cm, minimum height=1cm,text centered, draw=black]
%\tikzstyle{question} = [rectangle, minimum width=3cm, minimum height=1cm, text centered, draw=black]
%\tikzstyle{decision} = [trapezium, trapezium left angle=70, trapezium right angle=110, minimum width=3cm, minimum height=1cm, text centered, draw=black]
%\tikzstyle{question} = [rectangle, rounded corners, minimum width=3cm, minimum height=1cm,text centered, draw=black]
%\tikzstyle{process} = [rectangle, minimum width=3cm, minimum height=1cm, text centered, draw=black]
%\tikzstyle{decision} = [trapezium, trapezium left angle=70, trapezium right angle=110, minimum width=3cm, minimum height=1cm, text centered, draw=black]


\title{Practice}

\begin{document}
\begin{abstract}
  Try these problems.
\end{abstract}
\maketitle


\begin{exercise}
%{Define a vertical asymptote.}
%{Calculate limits of the form number over zero.}
%{Calculate limits of the form zero over zero.}
%\tag{limit}
Consider 
\[a(x) = \frac{x^2+x-6}{x^2-x-12}.
\]
Find all vertical asymptotes.
\begin{prompt}
\begin{multipleChoice}
\choice[correct]{There are vertical asymptotes}
\choice{There are no vertical asymptotes}
\end{multipleChoice}
\begin{exercise} 
\[
x=\answer{4}
\]
\end{exercise}
\end{prompt}
\end{exercise}

\begin{exercise}
%{Define a vertical asymptote.}
%{Calculate limits of the form number over zero.}
%{Calculate limits of the form zero over zero.}
%\tag{limit}
Consider 
\[s(x) = \frac{x^2-2 x-15}{x^2+3 x+2}.
\]
Find all vertical asymptotes.
\begin{prompt}
\begin{multipleChoice}
\choice[correct]{There are vertical asymptotes}
\choice{There are no vertical asymptotes}
\end{multipleChoice}
\begin{exercise}Write your answers from least to greatest:
\[
x=\answer{-2}\qquad\text{and}\qquad x=\answer{-1}
\]
\end{exercise}
\end{prompt}
\end{exercise}


\begin{exercise}

%\tag{derivative}

%{Identify where a function is and is not continuous.}

Give intervals on which each of the following functions are
continuous. Write combinations of intervals going from left to right
on the number line.

\begin{enumerate}
\item $\frac{1}{e^x+1}$ is continuous on $(\answer{-\infty},\answer{\infty})$.
\item $\frac{1}{x^2-1}$ is continuous on $(\answer{-\infty},\answer{-1})$ and $(\answer{-1},\answer{1})$ and $(\answer{1},\answer{\infty})$.
\item $\sqrt{5-x}$ is continuous on $(\answer{-\infty},\answer{5})$.
\item $\sqrt{5-x^2}$ is continuous on $(\answer{-\sqrt{5}},\answer{\sqrt{5}})$.
\end{enumerate}

\end{exercise}




\begin{exercise}

%{Calculate limits of piecewise functions.}
%{Identify where a function is and is not continuous.}

%\tag{piecewise}
%\tag{continuity}

Let
\[
f(x) =
\begin{cases}
  x^2-1, &x < 3 \\
  x+5,  &x\geq 3.
\end{cases}
\]
Is $f$ continuous everywhere? 

\begin{multipleChoice}
\choice[correct]{Yes}
\choice{No}
\end{multipleChoice}

\end{exercise}







\begin{exercise}

%{Calculate limits using the limit laws.}
%{Calculate limits of piecewise functions.}
%{Evaluate the limit as x approaches a point where there is a vertical asymptote.}
%%{Calculate the limit as x approaches $\pm\infty$ of common functions algebraically.}

%\tag{limits}
Let
\[
g(x)=\begin{cases}
-2e^{x} & \text{if $x<0$}\\
\frac{x+6}{x-3} & \text{if $x\ge0$}.
\end{cases}
\]

Find
\begin{enumerate}
\item		$\lim_{x\to 0^{-}} g(x)\begin{prompt} = \answer{-2}\end{prompt}$
\item		$\lim_{x\to 0^+} g(x)\begin{prompt} = \answer{-2}\end{prompt}$
\item		$\lim_{x\to 3^{-}} g(x)\begin{prompt} = \answer{-\infty}\end{prompt}$
\item		$\lim_{x\to 3^{+}} g(x)\begin{prompt} = \answer{\infty}\end{prompt}$
\item		$\lim_{x\to -\infty} g(x)\begin{prompt} = \answer{0}\end{prompt}$
\item		$\lim_{x\to +\infty} g(x)\begin{prompt} = \answer{1}\end{prompt}$
\end{enumerate}
\end{exercise}

\begin{exercise}
%{State the Intermediate Value Theorem including hypotheses.}
%\tag{continuity}
%\tag{intermediate value theorem}
  The Intermediate Value Theorem states:
  If $f$ is a continuous function for all $x$ in the closed interval
  $[a,b]$ and $r$ is between
  \wordChoice{\choice{a}\choice[correct]{f(a)}} and
  \wordChoice{\choice{b}\choice[correct]{f(b)}}, then there is a
  number \wordChoice{\choice[correct]{u}\choice{f(u)}} in
  $[a, b]$ such that
  \wordChoice{\choice[correct]{f(u) = r}\choice{f(r) = u}}.
  \begin{hint}
    Consider the following graph:
    \begin{image}
      \begin{tikzpicture}
        \begin{axis}[
            domain=0:6, ymin=0, ymax=2.2,xmax=6,
            axis lines =left, xlabel=$x$, ylabel=$y$,
            every axis y label/.style={at=(current axis.above origin),anchor=south},
            every axis x label/.style={at=(current axis.right of origin),anchor=west},
            xtick={1,3.597,5}, ytick={.203,1,1.679},
            xticklabels={$a$,$u$,$b$}, yticklabels={$f(a)$,$r$,$f(b)$},
            axis on top,
          ]
          \addplot [draw=none, fill=fill2, domain=(0:7)] {1.679} \closedcycle;
          \addplot [draw=none, fill=background, domain=(0:7)] {.203} \closedcycle;
          \addplot [textColor,dashed] plot coordinates {(0,1.679) (6,1.679)};
          \addplot [textColor,dashed] plot coordinates {(0,.203) (6,.203)};
          \addplot [textColor,dashed] plot coordinates {(5,0) (5,1.679)};
          \addplot [textColor,dashed] plot coordinates {(1,0) (1,.203)};
          \addplot [textColor,dashed] plot coordinates {(3.587,0) (3.597,1)};
          \addplot [penColor2,domain=(0:6)] {1};
          \addplot [very thick,penColor, smooth,domain=(0:2.5)] {sin(deg((x - 4)/2)) + 1.2};
          \addplot [very thick,penColor, smooth,domain=(4:6)] {sin(deg((x - 4)/2)) + 1.2};
          \addplot [very thick,dashed,penColor!50!background, smooth,domain=(2.5:4)] {sin(deg((x - 4)/2)) + 1.2}; 
          \addplot [color=penColor!50!background,fill=penColor!50!background,only marks,mark=*] coordinates{(3.587,1)};  %% closed hole          
          \addplot [color=penColor,fill=penColor,only marks,mark=*] coordinates{(1,.203)};  %% closed hole          
          \addplot [color=penColor,fill=penColor,only marks,mark=*] coordinates{(5,1.679)};  %% closed hole          
        \end{axis}
      \end{tikzpicture}
    \end{image}
  \end{hint}
\end{exercise}

\begin{exercise}
  
%{Understand the connection between continuity of a function and the value of a limit.}
%{Understand what it means for a function to be continuous.}

%\tag{continuity}

Is the function
\[
f(x)=\left\{\begin{array}{ccc} 
\frac{x^2-64}{x^2-11 x+24},		& & x\ne8\\
5, & & x=8
\end{array}\right.
\]
continuous at $x=0$ or $x=8$?

\begin{prompt}
\begin{multipleChoice}
\choice{$f$ is continuous at both $x=0$ and $x=8$.}
\choice[correct]{$f$ is continuous at $x=0$ but not at $x=8$.}
\choice{$f$ is continuous at at $x=8$ but not at $x=0$.}
\choice{$f$ is not continuous at $x=0$ and $x=8$.}
\end{multipleChoice}
\end{prompt}

\end{exercise}

\begin{exercise}

%{Explain why certain points exist using the Intermediate Value Theorem.}

%\tag{continuity}
%\tag{intermediate value theorem}

Let $f$ be continuous on $\left[1,5\right]$ where $f(1)=-2$ and $f(5)=-10$. Does a value $1<c<5$ exist such that $f(c)=-9$?

\begin{multipleChoice}
\choice{There does not exist a value.}
\choice[correct]{Yes, by the Intermediate Value Theorem}
\choice{Yes, by the Mean Value Theorem}
\choice{There does not necessarily exist such a value}
\end{multipleChoice}

\end{exercise}


\end{document}
