\documentclass{ximera}

\input{../preamble.tex}

\title[Dig-In:]{Integration With Initial Conditions}

\begin{document}
\begin{abstract}
  
\end{abstract}
\maketitle

Suppose we know that $f"$ is the derivative of some function $f$, is there a way to explicitly find $f$? The 
answer is yes, provided that we have some additional information: 

$$\int f' \d x = F(x)+C$$

Where $F(x)$ is some antiderivative of $f'(x)$ and $C$ is an arbitrary constant. The problem is that $F(x)$ and $f(x)$ are not
necessarily the same, just that they are both antiderivatives of $f'$. What we do know is they differ by a constant. For convenience, 
write

$$F(x)+C_{0}=f(x)$$ 

where $C_{0}$ is a specific constant. In particular:

$$C_{0} = F(x)-f(x)$$

If we know a particular value of $f(x)$ then we can find $C_{0}$. This particular value that we know is called an initial condition.

\begin{example}
Suppose that $f'(x)=3x^7$ and $f(1)= 2$, find $f(x)$.
\begin{explanation}
First we will find the indefinite integral of $f'$:
\begin{align*}
\int 8 x^7 \d x &= 3 \int x^7 \d x\\
&= 3 \cdot \answer[given]{\frac{x^8}{8}}+C.
\end{align*}
So we now know that 
$$f(x)=3 \cdot \answer[given]{\frac{x^8}{8}}+C.$$
Now we use the initial condition, $f(1)=2$ to find the specific value of $C$
$$2=f(1)=3 \cdot \answer[given]{\frac{1^8}{8}}+C$$
So
$$C=2-\frac{3}{8} = \answer[given]{\frac{13}{8}}$$
Therefore
$$f=\answer[given]{\frac{3x^{8}}{8}+\frac{13}{8}}$$
\end{explanation}
\end{example}

\section{Marginal Revenue and Marginal Cost}

Let's apply what we've learned about integrating with initial conditions to working with marginal revenue and marginal cost. 

Suppose that $R(q)$ is a total revenue function, then as money is earned only from selling products,
$$R(0)=0$$
Likewise, if $C(q)$ is a total cost function, then 
$$C(0)=C_{0}$$
where $C_{0}$ is the fixed cost. 

In both of these cases, we have a standard initial condition that we may use.

\begin{example}
Suppose that $R'(q)=500+3q^{2}$ is a marginal revenue function. Find the total revenue function $R(q)$.
\begin{explanation}
First we will find the indefinite integral of $R'$:
\begin{align*}
\int R'(q) \d q &= \int 500+3q^{2} \d q\\
&= \answer[given]{500q+q^{3}+C}
\end{align*}
So we now know that 
$$R(q)=500q+q^{3}+C$$
Now we use the initial condition, $R(0)=)$ to find the specific value of $C$
$$0=R(0)=0+C$$
So
$$C= \answer[given]{0}$$
Therefore
$$R(q)=500q+q^{3}$$
\end{explanation}
\end{example}

\begin{example}
Suppose that $C'(q)=200+q$ is a marginal cost function. Find the total cost function $C(q)$ if fixed costs are 1000.
\begin{explanation}
First we will find the indefinite integral of $C'$:
\begin{align*}
\int C'(q) \d q &= \int 200+q \d q\\
&= \answer[given]{200q+\frac{q^{2}}{2}+C}
\end{align*}
So we now know that 
$$R(q)=200q+\frac{q^{2}}{2}+C$$
Now we use the initial condition. Since the fixed costs are 1000, this means $C(0)=1000$. We can 
now use this to fnd the specific value of $C$
$$1000=C(0)=0+C$$
So
$$C= \answer[given]{1000}$$
Therefore
$$C(q)=200q+\frac{q^{2}}{2}+1000$$
\end{explanation}
\end{example}

\end{document}
