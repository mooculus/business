\documentclass{ximera}

\input{../preamble.tex}


\title[Break-Ground:]{How can we find how big a function can get}

\begin{document}
\begin{abstract}
Two young mathematicians race to math class.
\end{abstract}
\maketitle

Check out this dialogue between two calculus students (based on a true
story):

%% Brad suggest maybe make one constant rate

\begin{dialogue}
\item[Devyn] 
\item[Riley] \end{dialogue}

\begin{problem}
  Which of the following describes the race above?
  \begin{multipleChoice}
    \choice{Devyn was leading until the end, when the race finished in a tie.}
    \choice{Riley was leading until the end, when the race finished in a tie.}
    \choice{Devyn was leading, then Riley was leading until the end, when the race finished in a tie.}
    \choice[correct]{Riley was leading, then Devyn was leading until the end, when the race finished in a tie.}
    \choice{None of the above.}
  \end{multipleChoice}
\end{problem}

\begin{problem}
  What can you say about Devyn's and Riley's average velocities?
  \begin{multipleChoice}
    \choice{Devyn has the larger average velocity.}
    \choice{Riley has the larger average velocity.}
    \choice[correct]{Their average velocities are equal.}
    \choice{None of the above.}
  \begin{feedback}
    Since Devyn and Riley start and stop at the same time and place,
    their average velocities are equal.
  \end{feedback}
  \end{multipleChoice}
\end{problem}

\begin{problem}
  Record your guess to Riley's question: is there a moment during the race where Devyn and Riley were running at exactly the same speed?
  \begin{freeResponse}
  	Enter your guess of ``yes'' or ``no'', then come back after the Dig-In to see if your guess was correct!
  \end{freeResponse}
\end{problem}

%\input{../leveledQuestions.tex}


\end{document}
