\documentclass{ximera}

\input{../preamble}

\title[Dig-In:]{Financial Mathematics}





\begin{document}
\begin{abstract}
 \end{abstract}
\maketitle

\section{Basic Functions}
In financial mathematics, there are consumers and producers. Generally, consumers {\bf demand} items and producers {\bf supply} items. 

The price per item that consumers will demand and purchase $q$ items is the {\bf demand function} and is denoted by 
$$D(q)$$

Likewise, the price per item that producers will be willing produce and sell $q$ items is the {\bf supply function} and is denoted by
$$S(q)$$

The {\bf market equilibrium point} is a point $(q_{0},p_{0})$ where consumers are willing to pay $p_{0}$ for $q_{0}$ items at the
same time that producers are willing to set a price of $p_{0}$ for $q_{0}$ items. In other words:

$$D(q_{0})=p_{0}=S(q_{0})$$

So the point $(q_{0},p_{0})$ is the intersection point of the curves $S(q)$ and $D(q)$.



The {\bf (total) revenue} $R(q)$ from producing and selling $q$ items is related to the demand function and is given by
$$R(q)=qD(q)$$
In other words revenue is the product of quantity and price, where the price is the price determined by the demand function.

The {\bf (total) cost} of producing $q$ items is denoted
$$C(q)$$
Of note is the value
$$C(0)=C_{0}$$
This value is the {\bf fixed cost} of producing any number of items and does not change as $q$ changes.

The {\bf average cost} per item produced is
$$\overline{C}(q) = \frac{C}{q}$$

The {\bf (total) profit} $P(q)$ generated by producing and selling $q$ items is the difference between the revenue and cost
$$P(q)=R(q)-C(q)$$


\begin{question}
Suppose the demand function is $D(q)=100-3q$. What is the (total) revenue function $R(q)$?
\begin{selectAll}
\choice[correct]{$R(q)=100q-3q^{2}$}
\choice{$R(q)=100-2q$}
\choice{$R(q)=100q+3q^{2}$}
\choice{$R(q)=\frac{100-3q}{q}$}
\end{selectAll}
Suppose that $C(q)=200+3q+q^{2}$ is the (total) cost of producing and selling $q$ items. What is the average cost $\overline{C}(q)$ per item ?
\begin{selectAll}
\choice{$\overline{C}(q)=200+2q+q^{2}$}
\choice[correct]{$\overline{C}(q)=\frac{200+3q+q^{2}}{q}$}
\choice{$\overline{C}(q)=200q+2q^{2}+q^{3}$}
\choice{$\overline{C}(q)=200+2q+q^{3}$}
\end{selectAll}
\end{question}

\section{Compound Interest}

Compound interest is a type of exponential function which gives the value of an investment over time at a particular interest rate. 

Suppose:

$P \text{ is the principal (or initial investment)}$

$r \text{ is the {\bf annual} interest rate}$

$n \text{ is the number of compounding periods in one year}$

$t \text{ is the total number of years that this investment is made}$

$A \text{ is the value of this investment after } t \text{ years}$
then
$$ A = P\left( 1 + \frac{r}{n} \right)^{nt}$$







\end{document}

