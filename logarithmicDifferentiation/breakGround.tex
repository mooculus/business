\documentclass{ximera}

\input{../preamble.tex}



\title[Break-Ground:]{Multiplication to addition}

\begin{document}
\begin{abstract}
Two young mathematicians think about derivatives and logarithms.
\end{abstract}
\maketitle


Check out this dialogue between two calculus students (based on a true
story):

\begin{dialogue}
\item[Devyn] Riley, why is the product rule so much harder than the sum rule?
\item[Riley] Ever since 2nd grade, I've known that multiplication is
  \textit{harder} than addition.
\item[Devyn] I know! I was reading somewhere that a slide-rule somehow
  turns ``multiplication into addition.''
\item[Riley] Wow! I wonder how that works?
\item[Devyn] I \textit{think} it has something to do with logs?
\item[Riley] What? How does this work?
\end{dialogue}

Devyn is right, logarithms are used (and were invented) to convert
difficult multiplication problems into simpler addition problems.

\begin{problem}
  Let $f(x) = (x+3) \cdot (x-5) \cdot (x+8)$. Compute
  \[
  \ddx f(x)\begin{prompt} = \answer{(x-5)(x+8)+(x+3)(x-5)+(x+3)(x-5)}\end{prompt}
  \]
\end{problem}

Now, let's see what happens if we do the same problem but we take the
natural log of both sides first:
\begin{align*}
  f(x) &=(x+3) \cdot (x-5) \cdot (x+8)\\
  \ln(f(x)) &= \ln((x+3) \cdot (x-5) \cdot (x+8))\\
  \ln(f(x)) &=\ln(x+3) + \ln(x-4) + \ln(x+8)
\end{align*}

Now we'll take the derivative of both sides of the equation.
By the chain rule
\[
\ddx \ln(f(x))= \frac{f'(x)}{f(x)}
\]


\begin{problem}
  Compute %(using the chain rule)%
  \[
  \ddx \ln(x+3)  \begin{prompt}=\answer{\frac{1)}{x+3}}\end{prompt}
  \]
\end{problem}

\begin{problem}
  Compute %(using the chain rule)%
  \[
  \ddx \ln(x-5)  \begin{prompt}=\answer{\frac{1}{x-5}}\end{prompt}
  \]
\end{problem}

\begin{problem}
  Compute %(using the chain rule)%
  \[
  \ddx \ln(x+8)  \begin{prompt}=\answer{\frac{1}{x+8}}\end{prompt}
  \]
\end{problem}

So we have
\begin{align*}
  \frac{f'(x)}{f(x)} &= \frac{1}{x+3} + \frac{1}{x-5} + \frac{1}{x+8} \\
  f'(x) &= f(x) \left(\frac{1}{x+3} + \frac{1}{x-5} + \frac{1}{x+8}\right)\\
  &= (x+3)(x-5)(x+8)  \left(\frac{1}{x+3} + \frac{1}{x-5} + \frac{1}{x+8}\right)
\end{align*}



%\input{../leveledQuestions.tex}


\end{document}
