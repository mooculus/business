\documentclass{ximera}

\input{../preamble.tex}



\title[Break-Ground:]{Exponential Functions and Derivatives}

\begin{document}
\begin{abstract}

\end{abstract}
\maketitle

Check out this dialogue between two calculus students (based on a true
story):

\begin{dialogue}
\item[Devyn] We learned how to take the derivative of $x^n$, but I wonder how you find the derivative of $n^x$?
\item[Riley] That's easy, 
\[
\ddx n^x = xn^{x-1}
\]
\item[Devyn] I don't think it is that easy, but you could be right.
\item[Riley] Could be?
\end{dialogue}

To get an idea of how to find the derivative of an exponential function, let's look at one specific exponential function:
\[ 
f(x)=2^{x}
\]

Remember from the definition of the derivative that the derivative of a function at a value is approximated by slopes of secant lines. Let's 
look at what is happening with $f(x)=2^{x}$.

Recall that the slope of the secant line between $(x,f(x))$ and $(x+1,f(x+1))$ is 
\[
\frac{f(x+1)-f(x)}{x+1-x} = \frac{f(x+1)-f(x)}{1}
\]

{  \renewcommand*{\arraystretch}{1.3}
  \[
  \begin{array}{c|c|c}
    x & f(x) & \frac{f(x+1)-f(x)}{1}\\ \hline
    1 & 2 & 2\\
    2 & 4  & 4\\
    3 & 8  & 8 \\
    4 & 16 & 16
  \end{array}
  \]
  }



Looking at these, it seems that the derivative of $2^{x}$ is almost $2^{x}$. It turns out that it is a constant multiple of $2^{x}$. 

If we try the same 
process with an exponential function to a different base, we will see the same thing occur. Again, the derivative ends up being a constant 
multiple of the original function. 

What is this constant multiple? We will find out in the next few pages, but first, is there a basic exponential function 
where the constant is exactly 1? There is and we will look at this function next.


 
%\input{../leveledQuestions.tex}

\end{document}
