\documentclass{ximera}

%\usepackage{todonotes}

\newcommand{\todo}{}

%\usepackage{esint} % for \oiint
\graphicspath{
{./}
{functionsOfSeveralVariables/}
{normalVectors/}
{lagrangeMultipliers/}
{vectorFields/}
{greensTheorem/}
{shapeOfThingsToCome/}
}


\usepackage{tkz-euclide}

\tikzset{>=stealth} %% cool arrow head
\tikzset{shorten <>/.style={ shorten >=#1, shorten <=#1 } } %% allows shorter vectors

\usetikzlibrary{backgrounds} %% for boxes around graphs
\usetikzlibrary{shapes,positioning}  %% Clouds and stars
\usetikzlibrary{matrix} %% for matrix
\usetikzlibrary{patterns} 
\usepgfplotslibrary{polar} %% for polar plots

\usepackage[makeroom]{cancel} %% for strike outs
%\usepackage{mathtools} %% for pretty underbrace % Breaks Ximera
\usepackage{multicol}
\usepackage{pgffor} %% required for integral for loops


%% http://tex.stackexchange.com/questions/66490/drawing-a-tikz-arc-specifying-the-center
%% Draws beach ball
\tikzset{pics/carc/.style args={#1:#2:#3}{code={\draw[pic actions] (#1:#3) arc(#1:#2:#3);}}}



\usepackage{array}
\setlength{\extrarowheight}{+.1cm}   
\newdimen\digitwidth
\settowidth\digitwidth{9}
\def\divrule#1#2{
\noalign{\moveright#1\digitwidth
\vbox{\hrule width#2\digitwidth}}}





\newcommand{\RR}{\mathbb R}
\newcommand{\R}{\mathbb R}
\newcommand{\N}{\mathbb N}
\newcommand{\Z}{\mathbb Z}

%\newcommand{\sage}{\textsf{SageMath}}


%\renewcommand{\d}{\,d\!}
\renewcommand{\d}{\mathop{}\!d}
\newcommand{\dd}[2][]{\frac{\d #1}{\d #2}}
\newcommand{\pp}[2][]{\frac{\partial #1}{\partial #2}}
\renewcommand{\l}{\ell}
\newcommand{\ddx}{\frac{d}{\d x}}

\newcommand{\zeroOverZero}{\ensuremath{\boldsymbol{\tfrac{0}{0}}}}
\newcommand{\inftyOverInfty}{\ensuremath{\boldsymbol{\tfrac{\infty}{\infty}}}}
\newcommand{\zeroOverInfty}{\ensuremath{\boldsymbol{\tfrac{0}{\infty}}}}
\newcommand{\zeroTimesInfty}{\ensuremath{\small\boldsymbol{0\cdot \infty}}}
\newcommand{\inftyMinusInfty}{\ensuremath{\small\boldsymbol{\infty - \infty}}}
\newcommand{\oneToInfty}{\ensuremath{\boldsymbol{1^\infty}}}
\newcommand{\zeroToZero}{\ensuremath{\boldsymbol{0^0}}}
\newcommand{\inftyToZero}{\ensuremath{\boldsymbol{\infty^0}}}



\newcommand{\numOverZero}{\ensuremath{\boldsymbol{\tfrac{\#}{0}}}}
\newcommand{\dfn}{\textbf}
%\newcommand{\unit}{\,\mathrm}
\newcommand{\unit}{\mathop{}\!\mathrm}
\newcommand{\eval}[1]{\bigg[ #1 \bigg]}
\newcommand{\seq}[1]{\left( #1 \right)}
\renewcommand{\epsilon}{\varepsilon}
\renewcommand{\phi}{\varphi}


\renewcommand{\iff}{\Leftrightarrow}

\DeclareMathOperator{\arccot}{arccot}
\DeclareMathOperator{\arcsec}{arcsec}
\DeclareMathOperator{\arccsc}{arccsc}
\DeclareMathOperator{\si}{Si}
\DeclareMathOperator{\proj}{\vec{proj}}
\DeclareMathOperator{\scal}{scal}
\DeclareMathOperator{\sign}{sign}


%% \newcommand{\tightoverset}[2]{% for arrow vec
%%   \mathop{#2}\limits^{\vbox to -.5ex{\kern-0.75ex\hbox{$#1$}\vss}}}
\newcommand{\arrowvec}{\overrightarrow}
%\renewcommand{\vec}[1]{\arrowvec{\mathbf{#1}}}
\renewcommand{\vec}{\mathbf}
\newcommand{\veci}{{\boldsymbol{\hat{\imath}}}}
\newcommand{\vecj}{{\boldsymbol{\hat{\jmath}}}}
\newcommand{\veck}{{\boldsymbol{\hat{k}}}}
\newcommand{\vecl}{\boldsymbol{\l}}
\newcommand{\uvec}[1]{\mathbf{\hat{#1}}}
\newcommand{\utan}{\mathbf{\hat{t}}}
\newcommand{\unormal}{\mathbf{\hat{n}}}
\newcommand{\ubinormal}{\mathbf{\hat{b}}}

\newcommand{\dotp}{\bullet}
\newcommand{\cross}{\boldsymbol\times}
\newcommand{\grad}{\boldsymbol\nabla}
\newcommand{\divergence}{\grad\dotp}
\newcommand{\curl}{\grad\cross}
%\DeclareMathOperator{\divergence}{divergence}
%\DeclareMathOperator{\curl}[1]{\grad\cross #1}
\newcommand{\lto}{\mathop{\longrightarrow\,}\limits}

\renewcommand{\bar}{\overline}

\colorlet{textColor}{black} 
\colorlet{background}{white}
\colorlet{penColor}{blue!50!black} % Color of a curve in a plot
\colorlet{penColor2}{red!50!black}% Color of a curve in a plot
\colorlet{penColor3}{red!50!blue} % Color of a curve in a plot
\colorlet{penColor4}{green!50!black} % Color of a curve in a plot
\colorlet{penColor5}{orange!80!black} % Color of a curve in a plot
\colorlet{penColor6}{yellow!70!black} % Color of a curve in a plot
\colorlet{fill1}{penColor!20} % Color of fill in a plot
\colorlet{fill2}{penColor2!20} % Color of fill in a plot
\colorlet{fillp}{fill1} % Color of positive area
\colorlet{filln}{penColor2!20} % Color of negative area
\colorlet{fill3}{penColor3!20} % Fill
\colorlet{fill4}{penColor4!20} % Fill
\colorlet{fill5}{penColor5!20} % Fill
\colorlet{gridColor}{gray!50} % Color of grid in a plot

\newcommand{\surfaceColor}{violet}
\newcommand{\surfaceColorTwo}{redyellow}
\newcommand{\sliceColor}{greenyellow}




\pgfmathdeclarefunction{gauss}{2}{% gives gaussian
  \pgfmathparse{1/(#2*sqrt(2*pi))*exp(-((x-#1)^2)/(2*#2^2))}%
}


%%%%%%%%%%%%%
%% Vectors
%%%%%%%%%%%%%

%% Simple horiz vectors
\renewcommand{\vector}[1]{\left\langle #1\right\rangle}


%% %% Complex Horiz Vectors with angle brackets
%% \makeatletter
%% \renewcommand{\vector}[2][ , ]{\left\langle%
%%   \def\nextitem{\def\nextitem{#1}}%
%%   \@for \el:=#2\do{\nextitem\el}\right\rangle%
%% }
%% \makeatother

%% %% Vertical Vectors
%% \def\vector#1{\begin{bmatrix}\vecListA#1,,\end{bmatrix}}
%% \def\vecListA#1,{\if,#1,\else #1\cr \expandafter \vecListA \fi}

%%%%%%%%%%%%%
%% End of vectors
%%%%%%%%%%%%%

%\newcommand{\fullwidth}{}
%\newcommand{\normalwidth}{}



%% makes a snazzy t-chart for evaluating functions
%\newenvironment{tchart}{\rowcolors{2}{}{background!90!textColor}\array}{\endarray}

%%This is to help with formatting on future title pages.
\newenvironment{sectionOutcomes}{}{} 



%% Flowchart stuff
%\tikzstyle{startstop} = [rectangle, rounded corners, minimum width=3cm, minimum height=1cm,text centered, draw=black]
%\tikzstyle{question} = [rectangle, minimum width=3cm, minimum height=1cm, text centered, draw=black]
%\tikzstyle{decision} = [trapezium, trapezium left angle=70, trapezium right angle=110, minimum width=3cm, minimum height=1cm, text centered, draw=black]
%\tikzstyle{question} = [rectangle, rounded corners, minimum width=3cm, minimum height=1cm,text centered, draw=black]
%\tikzstyle{process} = [rectangle, minimum width=3cm, minimum height=1cm, text centered, draw=black]
%\tikzstyle{decision} = [trapezium, trapezium left angle=70, trapezium right angle=110, minimum width=3cm, minimum height=1cm, text centered, draw=black]



\begin{document}
\begin{abstract}
We see that if a function is differentiable at a point, then it must
be continuous at that point.
\end{abstract}
\maketitle

There are connections between continuity and differentiability.

\begin{theorem}[Differentiability Implies Continuity]\index{differentiability implies continuity}
If $f$ is a differentiable function at $x = a$, then $f$ is continuous
at $x=a$.
\begin{explanation}
To explain why this is true, we are going to use the following
definition of the derivative
\[
f'(a) = \lim_{x\to a} \frac{f(x)-f(a)}{x-a}.
\]

  Assuming that $f'(a)$ exists, we want to show that $f(x)$ is
continuous at $x=a$, hence we must show that
\[
\lim_{x\to a} f(x) = f(a).
\]
Starting with
\[
\lim_{x\to a} \left(f(x) - f(a)\right)
\]
we multiply and divide by $(x-a)$ to get
\begin{align*}
  &= \lim_{x\to a} \left((x-a)\frac{f(x) - f(a)}{x-a}\right) \\
  &= \left(\lim_{x\to a} (x-a) \right) \left(\lim_{x\to a}\frac{f(x) - f(a)}{x-a}\right) &\text{Limit Law.} \\
  &= \answer[given]{0}\cdot f'(a) = \answer[given]{0}.
\end{align*}
Since 
\[
\lim_{x\to a}\left(f(x) - f(a)\right) = 0 
\]
we see that $\lim_{x\to a} f(x) = f(a)$, and so $f$ is continuous at
$x=a$.
\end{explanation}
\end{theorem}

This theorem is often written as its contrapositive:
\begin{quote}
If $f(x)$ is not continuous at $x=a$, then $f(x)$ is not
differentiable at $x=a$.
\end{quote}


Thus from the theorem above, we see that all differentiable functions
on $\RR$ are continuous on $\RR$. Nevertheless there are continuous
functions on $\RR$ that are not differentiable on $\RR$.

\begin{question}
  Which of the following functions are continuous but not
  differentiable on $\RR$?
  \begin{multipleChoice}
    \choice{$x^2$}
    \choice{$\lfloor x \rfloor$}
    \choice[correct]{$|x|$}
    \choice{$\frac{e^{x}}{x}$}
  \end{multipleChoice}
\end{question}

\begin{example}
  Consider
  \[
  f(x) = \begin{cases}
          x^2 &\text{if $x<3$,}\\
          mx+b &\text{if $x\ge 3$.}
         \end{cases}
  \]
  What values of $m$ and $b$ make $f$ differentiable at $x=3$?
  \begin{explanation}
    To start, we know that we must make $f$ both continuous and
    differentiable. We will start by showing $f$ is continuous at
    $x=3$. Write with me:
    \begin{align*}
      \lim_{x\to 3^-} f(x) &= \answer[given]{9}\\
      \lim_{x\to 3^+} f(x) &= \answer[given]{m 3 + b}\\
      f(3) &= \answer[given]{m 3 + b}
    \end{align*}
    So for the function to be continuous, we must have
    \[
    m\cdot 3 + b =9.
    \]
    We also must ensure that the value of the derivatives of both
    pieces of $f$ agree at $x=3$. Write with me
    \begin{align*}
      \ddx x^2 &= \lim_{h\to 0}\frac{(x+h)^2-x^2}{h}\\
      &= \lim_{h\to 0}\frac{x^2 + 2xh + h^2 - x^2}{h}\\
      &= \lim_{h\to 0}\frac{2xh + h^2}{h}\\
      &= \lim_{h\to 0}\left(2x + h\right)\\
      &=2x.
    \end{align*}
    Moreover, by the definition of a tangent line
    \[
    \ddx(mx+ b) = \answer[given]{m}
    \]
    Hence we must have
    \begin{align*}
      \eval{\ddx x^2}_{x=3} &= \eval{\ddx(mx+b)}_{x=3}\\
      \eval{2x}_{x=3} &= \eval{m}_{x=3}\\
      6 &= m.
    \end{align*}
    Ah! So now
    \begin{align*}
      9 &= m\cdot 3 + b\\
      9 &= 6\cdot 3 + b\\
      9 &= 18 + b,
    \end{align*}
    so $b=\answer[given]{-9}$. Thus setting $m=\answer[given]{6}$ and
    $b=\answer[given]{-9}$ will give us a differentiable (and hence
    continuous) piecewise function.
    \begin{onlineOnly}
      We can confirm our results by looking at the graph of $y=f(x)$:
      \[
      \graph[xmin=0,xmax=6,ymin=-1,ymax=15]{y=x^2\left\{x<3\right\},y=mx+b\left\{3 \leq x\right\},m=6,b=-9} 
      \]
    \end{onlineOnly}
  \end{explanation}
\end{example}


\end{document}
