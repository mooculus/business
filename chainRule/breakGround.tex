\documentclass{ximera}

\input{../preamble.tex}

\title[Break-Ground:]{A coincidence?}

\begin{document}
\begin{abstract}

\end{abstract}
\maketitle


Check out this dialogue between two calculus students (based on a true
story):

\begin{dialogue}
\item[Devyn] Yo Riley, I was looking at how the product rule works with more than two functions
  \item[Riley] What do you mean?
\item[Devyn] Well, look at this:
\[  
	\dd{x} f(x)g(x) h(x)  = f'(x) g(x) h(x) + f(x)  g'(x)  h(x) + f(x)  g(x)  h'(x) 
\]
 \item[Riley] So?
\item[Devyn] But, if $f(x)=g(x)=h(x)$, then each of the terms looks the same:
\[  
	\dd{x} f(x) f(x)  f(x)  = 3 f'(x)  f(x)  f(x) 
 \]
 \item[Riley] Where are you going with this?
 \item[Devyn] Well, I wonder what would happen if you tried to find the derivative of the product of many copies of the same function, would it be the same?
 \item[Riley] I'm not sure?
 \item[Devyn] Did you notice this:
\[ 
	f(x) f(x) f(x) = [f(x)]^3
\]
 and
\begin{align*}
\ddx f(x) f(x) f(x) &= \ddx [f(x)]^3\\ 
&=  3 f'(x)  f(x)  f(x) \\
&= 3 [f(x)]^2 f'(x) 
\end{align*}
 \item[Riley] So?
 \item[Devyn] Do you think there is some way to use the derivative of $x^3$ to do this problem?
  \item[Riley] Maybe? Maybe there is a formula for this, just like there seems to be a formula for everything?

\end{dialogue}


Let's look at the following function and try to compute its derivative:

$$f(x)=(x^{3})^{5}$$

Based on what we have seen so far, perhaps we should just perform some algebra first before finding the derivative:

\begin{align*}
f(x) &= (x^{3})^{5}\\ 
&=x^{15} \\
\end{align*}

So

\begin{align*}
\ddx f(x) &= \ddx x^{15}\\ 
&=15x^{14} \\
\end{align*}

Now, let's make a different computation:

$$\ddx x^{3} = 3x^{2}$$
and
$$\ddx x^{5} = 5x^{4}$$ 

So

\begin{align*}
5(x^{3})^{4} \cdot  3x^{2} &= 5x^{12} \cdot 3x^{2}\\ 
&=15x^{14} \\
\end{align*}

In other words, if $g(x)=x^{5}$ and $h(x)=x^{3}$, then

$$\ddx g(h(x))=g'(h(x)) \cdot h'(x)$$

Is this just a coincidence? No, it is not a coincidence. It turns out that this is a consequence of another rule of differentiation: The Chain Rule.

%%\input{../leveledQuestions.tex}


\end{document}
