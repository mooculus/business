\documentclass{ximera}

\input{../preamble.tex}

\outcome{Use the first derivative to determine whether a function is increasing or decreasing.}
\outcome{Define higher order derivatives.}
\outcome{Compare differing notations for higher order derivatives.}
\outcome{Identify the relationships between the function and its first and second derivatives.}


\title[Dig-In:]{Second derivative test}

\begin{document}
\begin{abstract}
 Here we look at the second derivative test.   
\end{abstract}
\maketitle

\section{Using the second derivative to locate extrema}


Recall the first derivative test:
\begin{itemize}
\item If $f'(x)>0$ to the left of $a$ and $f'(x)<0$ to the right of
  $a$, then $f(a)$ is a local maximum.
\item If $f'(x)<0$ to the left of $a$ and $f'(x)>0$ to the right of
  $a$, then $f(a)$ is a local minimum.
\end{itemize}

If $f'$ changes from positive to negative it is decreasing. In this
case, $f''$ might be negative, and if in fact $f''$ is negative
then $f'$ is definitely decreasing, so there is a local maximum at
the point in question. On the other hand, if $f'$ changes from
negative to positive it is increasing. Again, this means that
$f''$ might be positive, and if in fact $f''$ is positive then
$f'$ is definitely increasing, so there is a local minimum at the
point in question. We summarize this as the \textit{second derivative
  test}.

\begin{theorem}[Second Derivative Test]\index{second derivative test}\label{T:sdt}
Suppose that $f''(x)$ is continuous on an open interval and that
$f'(a)=0$ for some value of $a$ in that interval.
\begin{itemize}
\item If $f''(a) <0$, then $f$ has a local maximum at $a$.
\item If $f''(a) >0$, then $f$ has a local minimum at $a$.
\item If $f''(a) =0$, then the test is inconclusive. In this case,
  $f$ may or may not have a local extremum at $x=a$.
\end{itemize}
\end{theorem}


The second derivative test is often the easiest way to identify local
maximum and minimum points. Sometimes the test fails and sometimes
the second derivative is quite difficult to evaluate. In such cases we
must fall back on one of the previous tests.

\begin{example}
Once again, consider the function 
\[
f(x) = \frac{x^4}{4}+\frac{x^3}{3}-x^2
\]
Use the second derivative test, to locate the
local extrema of $f$.

\begin{explanation}
Start by computing
\[
f'(x) = \answer[given]{x^3 + x^2 -2x} \qquad\text{and}\qquad f''(x) = \answer[given]{3x^2 + 2x-2}.
\] 
Using the same technique as we used before, we find that 
\[
f'(-2) = \answer[given]{0},\qquad f'(0) = \answer[given]{0}, \qquad f'(1) = \answer[given]{0}. 
\]
Now we'll attempt to use the second derivative test,
\[
f''(-2) = \answer[given]{6}, \qquad f''(0) =\answer[given]{ -2}, \qquad f''(1) = \answer[given]{3}.
\]
Hence we see that $f$ has a local minimum at $x=-2$, a local
maximum at $x=0$, and a local minimum at $x=1$, see below for a plot
of $f(x) =x^4/4 + x^3/3 -x^2$ and $f''(x) = 3x^2 + 2x -2$:
\begin{image}
\begin{tikzpicture}
	\begin{axis}[
            domain=-4:4,
            ymax=7,
            ymin=-4,
            %samples=100,
            axis lines =middle, xlabel=$x$, ylabel=$y$,
            every axis y label/.style={at=(current axis.above origin),anchor=south},
            every axis x label/.style={at=(current axis.right of origin),anchor=west}
          ]
          \addplot [dashed, textColor, smooth] plot coordinates {(-2,-2.667) (-2,6)}; %% {.451};
          \addplot [dashed, textColor, smooth] plot coordinates {(1,0) (1,3)}; %% axis{2.215};

          \addplot [very thick, penColor, smooth] {(x^4)/4 + (x^3)/3 -x^2};
          \addplot [very thick, penColor4, smooth] {3*x^2 + 2*x -2};

          \node at (axis cs:-1.7,-2.7) [anchor=west] {\color{penColor}$f$};  
          \node at (axis cs:-1.5,2) [anchor=west] {\color{penColor4}$f''$};

          \addplot[color=penColor4,fill=penColor4,only marks,mark=*] coordinates{(-2,6)};  %% closed hole
          \addplot[color=penColor4,fill=penColor4,only marks,mark=*] coordinates{(1,3)};  %% closed hole
          \addplot[color=penColor4,fill=penColor4,only marks,mark=*] coordinates{(0,-2)};  %% closed hole
          \addplot[color=penColor,fill=penColor,only marks,mark=*] coordinates{(0,0)};  %% closed hole
          \addplot[color=penColor,fill=penColor,only marks,mark=*] coordinates{(-2,.-2.667)};  %% closed hole
          \addplot[color=penColor,fill=penColor,only marks,mark=*] coordinates{(1,-.4167)};  %% closed hole
        \end{axis}
\end{tikzpicture}
%% \caption{A plot of $f(x) =x^4/4 + x^3/3 -x^2$ and $f''(x) = 3x^2 + 2x -2$.}
%% \label{figure:SDT(x^4)/4 + (x^3)/3 -x^2}
\end{image}

\end{explanation}
\end{example}


\begin{question}
  If $f''(a)=0$, what does the second derivative test tell us?
  \begin{multipleChoice}
    \choice{The function has a local extrema at $x=a$.}
    \choice{The function does not have a local extrema at $x=a$.}
    \choice[correct]{It gives no information on whether $x=a$ is a local extremum.} 
  \end{multipleChoice}
  
\end{question}



\end{document}
