\documentclass{ximera}

%\usepackage{todonotes}

\newcommand{\todo}{}

%\usepackage{esint} % for \oiint
\graphicspath{
{./}
{functionsOfSeveralVariables/}
{normalVectors/}
{lagrangeMultipliers/}
{vectorFields/}
{greensTheorem/}
{shapeOfThingsToCome/}
}


\usepackage{tkz-euclide}

\tikzset{>=stealth} %% cool arrow head
\tikzset{shorten <>/.style={ shorten >=#1, shorten <=#1 } } %% allows shorter vectors

\usetikzlibrary{backgrounds} %% for boxes around graphs
\usetikzlibrary{shapes,positioning}  %% Clouds and stars
\usetikzlibrary{matrix} %% for matrix
\usetikzlibrary{patterns} 
\usepgfplotslibrary{polar} %% for polar plots

\usepackage[makeroom]{cancel} %% for strike outs
%\usepackage{mathtools} %% for pretty underbrace % Breaks Ximera
\usepackage{multicol}
\usepackage{pgffor} %% required for integral for loops


%% http://tex.stackexchange.com/questions/66490/drawing-a-tikz-arc-specifying-the-center
%% Draws beach ball
\tikzset{pics/carc/.style args={#1:#2:#3}{code={\draw[pic actions] (#1:#3) arc(#1:#2:#3);}}}



\usepackage{array}
\setlength{\extrarowheight}{+.1cm}   
\newdimen\digitwidth
\settowidth\digitwidth{9}
\def\divrule#1#2{
\noalign{\moveright#1\digitwidth
\vbox{\hrule width#2\digitwidth}}}





\newcommand{\RR}{\mathbb R}
\newcommand{\R}{\mathbb R}
\newcommand{\N}{\mathbb N}
\newcommand{\Z}{\mathbb Z}

%\newcommand{\sage}{\textsf{SageMath}}


%\renewcommand{\d}{\,d\!}
\renewcommand{\d}{\mathop{}\!d}
\newcommand{\dd}[2][]{\frac{\d #1}{\d #2}}
\newcommand{\pp}[2][]{\frac{\partial #1}{\partial #2}}
\renewcommand{\l}{\ell}
\newcommand{\ddx}{\frac{d}{\d x}}

\newcommand{\zeroOverZero}{\ensuremath{\boldsymbol{\tfrac{0}{0}}}}
\newcommand{\inftyOverInfty}{\ensuremath{\boldsymbol{\tfrac{\infty}{\infty}}}}
\newcommand{\zeroOverInfty}{\ensuremath{\boldsymbol{\tfrac{0}{\infty}}}}
\newcommand{\zeroTimesInfty}{\ensuremath{\small\boldsymbol{0\cdot \infty}}}
\newcommand{\inftyMinusInfty}{\ensuremath{\small\boldsymbol{\infty - \infty}}}
\newcommand{\oneToInfty}{\ensuremath{\boldsymbol{1^\infty}}}
\newcommand{\zeroToZero}{\ensuremath{\boldsymbol{0^0}}}
\newcommand{\inftyToZero}{\ensuremath{\boldsymbol{\infty^0}}}



\newcommand{\numOverZero}{\ensuremath{\boldsymbol{\tfrac{\#}{0}}}}
\newcommand{\dfn}{\textbf}
%\newcommand{\unit}{\,\mathrm}
\newcommand{\unit}{\mathop{}\!\mathrm}
\newcommand{\eval}[1]{\bigg[ #1 \bigg]}
\newcommand{\seq}[1]{\left( #1 \right)}
\renewcommand{\epsilon}{\varepsilon}
\renewcommand{\phi}{\varphi}


\renewcommand{\iff}{\Leftrightarrow}

\DeclareMathOperator{\arccot}{arccot}
\DeclareMathOperator{\arcsec}{arcsec}
\DeclareMathOperator{\arccsc}{arccsc}
\DeclareMathOperator{\si}{Si}
\DeclareMathOperator{\proj}{\vec{proj}}
\DeclareMathOperator{\scal}{scal}
\DeclareMathOperator{\sign}{sign}


%% \newcommand{\tightoverset}[2]{% for arrow vec
%%   \mathop{#2}\limits^{\vbox to -.5ex{\kern-0.75ex\hbox{$#1$}\vss}}}
\newcommand{\arrowvec}{\overrightarrow}
%\renewcommand{\vec}[1]{\arrowvec{\mathbf{#1}}}
\renewcommand{\vec}{\mathbf}
\newcommand{\veci}{{\boldsymbol{\hat{\imath}}}}
\newcommand{\vecj}{{\boldsymbol{\hat{\jmath}}}}
\newcommand{\veck}{{\boldsymbol{\hat{k}}}}
\newcommand{\vecl}{\boldsymbol{\l}}
\newcommand{\uvec}[1]{\mathbf{\hat{#1}}}
\newcommand{\utan}{\mathbf{\hat{t}}}
\newcommand{\unormal}{\mathbf{\hat{n}}}
\newcommand{\ubinormal}{\mathbf{\hat{b}}}

\newcommand{\dotp}{\bullet}
\newcommand{\cross}{\boldsymbol\times}
\newcommand{\grad}{\boldsymbol\nabla}
\newcommand{\divergence}{\grad\dotp}
\newcommand{\curl}{\grad\cross}
%\DeclareMathOperator{\divergence}{divergence}
%\DeclareMathOperator{\curl}[1]{\grad\cross #1}
\newcommand{\lto}{\mathop{\longrightarrow\,}\limits}

\renewcommand{\bar}{\overline}

\colorlet{textColor}{black} 
\colorlet{background}{white}
\colorlet{penColor}{blue!50!black} % Color of a curve in a plot
\colorlet{penColor2}{red!50!black}% Color of a curve in a plot
\colorlet{penColor3}{red!50!blue} % Color of a curve in a plot
\colorlet{penColor4}{green!50!black} % Color of a curve in a plot
\colorlet{penColor5}{orange!80!black} % Color of a curve in a plot
\colorlet{penColor6}{yellow!70!black} % Color of a curve in a plot
\colorlet{fill1}{penColor!20} % Color of fill in a plot
\colorlet{fill2}{penColor2!20} % Color of fill in a plot
\colorlet{fillp}{fill1} % Color of positive area
\colorlet{filln}{penColor2!20} % Color of negative area
\colorlet{fill3}{penColor3!20} % Fill
\colorlet{fill4}{penColor4!20} % Fill
\colorlet{fill5}{penColor5!20} % Fill
\colorlet{gridColor}{gray!50} % Color of grid in a plot

\newcommand{\surfaceColor}{violet}
\newcommand{\surfaceColorTwo}{redyellow}
\newcommand{\sliceColor}{greenyellow}




\pgfmathdeclarefunction{gauss}{2}{% gives gaussian
  \pgfmathparse{1/(#2*sqrt(2*pi))*exp(-((x-#1)^2)/(2*#2^2))}%
}


%%%%%%%%%%%%%
%% Vectors
%%%%%%%%%%%%%

%% Simple horiz vectors
\renewcommand{\vector}[1]{\left\langle #1\right\rangle}


%% %% Complex Horiz Vectors with angle brackets
%% \makeatletter
%% \renewcommand{\vector}[2][ , ]{\left\langle%
%%   \def\nextitem{\def\nextitem{#1}}%
%%   \@for \el:=#2\do{\nextitem\el}\right\rangle%
%% }
%% \makeatother

%% %% Vertical Vectors
%% \def\vector#1{\begin{bmatrix}\vecListA#1,,\end{bmatrix}}
%% \def\vecListA#1,{\if,#1,\else #1\cr \expandafter \vecListA \fi}

%%%%%%%%%%%%%
%% End of vectors
%%%%%%%%%%%%%

%\newcommand{\fullwidth}{}
%\newcommand{\normalwidth}{}



%% makes a snazzy t-chart for evaluating functions
%\newenvironment{tchart}{\rowcolors{2}{}{background!90!textColor}\array}{\endarray}

%%This is to help with formatting on future title pages.
\newenvironment{sectionOutcomes}{}{} 



%% Flowchart stuff
%\tikzstyle{startstop} = [rectangle, rounded corners, minimum width=3cm, minimum height=1cm,text centered, draw=black]
%\tikzstyle{question} = [rectangle, minimum width=3cm, minimum height=1cm, text centered, draw=black]
%\tikzstyle{decision} = [trapezium, trapezium left angle=70, trapezium right angle=110, minimum width=3cm, minimum height=1cm, text centered, draw=black]
%\tikzstyle{question} = [rectangle, rounded corners, minimum width=3cm, minimum height=1cm,text centered, draw=black]
%\tikzstyle{process} = [rectangle, minimum width=3cm, minimum height=1cm, text centered, draw=black]
%\tikzstyle{decision} = [trapezium, trapezium left angle=70, trapezium right angle=110, minimum width=3cm, minimum height=1cm, text centered, draw=black]


\title[Dig-In:]{Applied optimization}

\begin{document}
\begin{abstract}
  Now we put our optimization skills to work.
\end{abstract}
\maketitle


In this section, we will present several worked examples of
optimization problems. Our method for solving these problems is
essentially the following:
\begin{description}
\item[Draw a picture.] If possible, draw a schematic picture with all the relevant information. 
\item[Determine your goal.] We need identify what needs to be
  optimized.
\item[Find constraints.] What limitations are set on our
  optimization?
\item[Solve for a single variable.] Now you should have a function to optimize.
\item[Use calculus to find the extreme values.] Be sure to check your answer!
\end{description}






\begin{example}
You are making cylindrical containers to contain a given volume.  Suppose
that the top and bottom are made of a material that is $N$ times as
expensive (cost per unit area) as the material used for the lateral side of
the cylinder.  Find (in terms of $N$) the ratio of height to base radius of
the cylinder that minimizes the cost of making the containers.
\begin{explanation}
  First we draw a picture:
\begin{image}
\begin{tikzpicture}
\draw[penColor,very thick] (0,2) ellipse (2 and .7);
\draw[very thick,penColor!20!background] (2,-2) arc (0:180:2 and .7);% top half of ellipse
\draw[very thick,penColor] (-2,-2) arc (180:360:2 and .7);% bottom half of ellipse

\draw[penColor, very thick] (2,2) -- (2,-2);
\draw[penColor, very thick] (-2,2) -- (-2,-2);

\draw[penColor, dashed, very thick] (0,2) -- (2,2);
\draw[penColor, dashed, very thick] (0,2) -- (0,-2);

\node [above,penColor] at (1,2) {$r$};
\node [left,penColor] at (0,-.33) {$h$};
\node [penColor,right] at (2,-1.42) {$V = \pi r^2h$};
\end{tikzpicture}
\end{image}
  Letting $c$ represent the cost of the lateral side, we can write an
  expression for the cost of materials:
  \[
  C = 2\pi crh+\answer[given]{2\pi r^2Nc}.
  \]
  Since we know that $V=\pi r^2h$, we can use this relationship to
  eliminate $h$ (we could eliminate $r$, but it's a little easier if we
  eliminate $h$, which appears in only one place in the above formula
  for cost).  We find
\begin{align*}
  C(r)&=2c\pi r\frac{V}{\pi r^2}+2Nc\pi r^2\\
  &=\frac{2cV}{r}+2Nc\pi r^2.
\end{align*}
We want to know the minimum value of this function when $r$ is in
$(0,\infty)$. Setting
\[
C'(r)=\answer[given]{-2cV/r^2+4Nc\pi r} =0
\]
we find $r=\sqrt[3]{V/(2N\pi)}$.  Since $C''(r)=\answer[given]{4cV/r^3+4Nc\pi}$ is
positive when $r$ is positive, there is a local minimum at the
critical value, and hence a global minimum since there is only one
critical value.

Finally, since $h=V/(\pi r^2)$, 
\begin{align*}
\frac{h}{r}&=\frac{V}{\pi r^3}\\ 
&=\frac{V}{\pi(V/(2N\pi))}\\ 
&=\answer[given]{2N},
\end{align*}
so the minimum cost occurs when the height $h$ is $2N$ times the
radius. If, for example, there is no difference in the cost of
materials, the height is twice the radius.
\end{explanation}
\end{example}






\begin{example}
  You want to sell a certain number $n$ of items in order to maximize
  your profit.  Market research tells you that if you set the price at
  \$$1.50$, you will be able to sell $5000$ items, and for every $10$
  cents you lower the price below \$$1.50$ you will be able to sell
  another $1000$ items.  Suppose that your fixed costs (``start-up
  costs'') total \$$2000$, and the per item cost of production
  (``marginal cost'') is \$$0.50$.  Find the price to set per item and
  the number of items sold in order to maximize profit, and also
  determine the maximum profit you can get.
\begin{explanation}
The first step is to convert the problem into a function maximization
problem. The revenue for selling $n$ items at $x$ dollars is given by
\[
r(x) = nx
\]
and the cost of producing $n$ items is given by
\[
c(x) = 2000+0.5 n. 
\]
However, from the problem we see that the number of items sold is
itself a function of $x$,
\[
n(x) =5000+\frac{1000(1.5-x)}{0.10}
\]
So profit is give by:
\begin{align*}
P(x) &= r(x) - c(x)\\
&= nx - (2000+0.5 n)\\
&=-10000x^2+25000x-12000. 
\end{align*}
We want to know the maximum value of this function when $x$ is between
0 and $1.5$. The derivative is
\[
P'(x)=\answer[given]{-20000x+25000},
\]
which is zero when $x=1.25$. Since $P''(x)=-20000<0$, there must be a
local maximum at $x=1.25$, and since this is the only critical value
it must be a global maximum as well. Alternately, we could compute
$P(0)=-12000$, $P(1.25)=3625$, and $P(1.5)=3000$ and note that
$P(1.25)$ is the maximum of these. Thus the maximum profit is
\$$3625$, attained when we set the price at \$$1.25$ and sell $7500$
items.
\begin{onlineOnly} 
   We can confirm our results by looking at the graph of $y=P(x)$:
   \[
   \graph[xmin=0,xmax=3,ymin=-4000,ymax=6000]{y=-10000x^2+25000x-12000}
   \]
\end{onlineOnly}
\end{explanation}
\end{example}








\end{document}
