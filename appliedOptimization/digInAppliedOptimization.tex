\documentclass{ximera}

\input{../preamble.tex}

\title[Dig-In:]{Applied optimization}

\begin{document}
\begin{abstract}
  Now we put our optimization skills to work.
\end{abstract}
\maketitle


In this section, we will present several worked examples of
optimization problems. Our method for solving these problems is
essentially the following:
\begin{description}
\item[Draw a picture.] If possible, draw a schematic picture with all the relevant information. 
\item[Determine your goal.] We need identify what needs to be
  optimized.
\item[Find constraints.] What limitations are set on our
  optimization?
\item[Solve for a single variable.] Now you should have a function to optimize.
\item[Use calculus to find the extreme values.] Be sure to check your answer!
\end{description}






\begin{example}
You are making cylindrical containers to contain a given volume.  Suppose
that the top and bottom are made of a material that is $N$ times as
expensive (cost per unit area) as the material used for the lateral side of
the cylinder.  Find (in terms of $N$) the ratio of height to base radius of
the cylinder that minimizes the cost of making the containers.
\begin{explanation}
  First we draw a picture:
\begin{image}
\begin{tikzpicture}
\draw[penColor,very thick] (0,2) ellipse (2 and .7);
\draw[very thick,penColor!20!background] (2,-2) arc (0:180:2 and .7);% top half of ellipse
\draw[very thick,penColor] (-2,-2) arc (180:360:2 and .7);% bottom half of ellipse

\draw[penColor, very thick] (2,2) -- (2,-2);
\draw[penColor, very thick] (-2,2) -- (-2,-2);

\draw[penColor, dashed, very thick] (0,2) -- (2,2);
\draw[penColor, dashed, very thick] (0,2) -- (0,-2);

\node [above,penColor] at (1,2) {$r$};
\node [left,penColor] at (0,-.33) {$h$};
\node [penColor,right] at (2,-1.42) {$V = \pi r^2h$};
\end{tikzpicture}
\end{image}
  Letting $c$ represent the cost of the lateral side, we can write an
  expression for the cost of materials:
  \[
  C = 2\pi crh+\answer[given]{2\pi r^2Nc}.
  \]
  Since we know that $V=\pi r^2h$, we can use this relationship to
  eliminate $h$ (we could eliminate $r$, but it's a little easier if we
  eliminate $h$, which appears in only one place in the above formula
  for cost).  We find
\begin{align*}
  C(r)&=2c\pi r\frac{V}{\pi r^2}+2Nc\pi r^2\\
  &=\frac{2cV}{r}+2Nc\pi r^2.
\end{align*}
We want to know the minimum value of this function when $r$ is in
$(0,\infty)$. Setting
\[
C'(r)=\answer[given]{-2cV/r^2+4Nc\pi r} =0
\]
we find $r=\sqrt[3]{V/(2N\pi)}$.  Since $C''(r)=\answer[given]{4cV/r^3+4Nc\pi}$ is
positive when $r$ is positive, there is a local minimum at the
critical value, and hence a global minimum since there is only one
critical value.

Finally, since $h=V/(\pi r^2)$, 
\begin{align*}
\frac{h}{r}&=\frac{V}{\pi r^3}\\ 
&=\frac{V}{\pi(V/(2N\pi))}\\ 
&=\answer[given]{2N},
\end{align*}
so the minimum cost occurs when the height $h$ is $2N$ times the
radius. If, for example, there is no difference in the cost of
materials, the height is twice the radius.
\end{explanation}
\end{example}






\begin{example}
  You want to sell a certain number $n$ of items in order to maximize
  your profit.  Market research tells you that if you set the price at
  \$$1.50$, you will be able to sell $5000$ items, and for every $10$
  cents you lower the price below \$$1.50$ you will be able to sell
  another $1000$ items.  Suppose that your fixed costs (``start-up
  costs'') total \$$2000$, and the per item cost of production
  (``marginal cost'') is \$$0.50$.  Find the price to set per item and
  the number of items sold in order to maximize profit, and also
  determine the maximum profit you can get.
\begin{explanation}
The first step is to convert the problem into a function maximization
problem. The revenue for selling $n$ items at $x$ dollars is given by
\[
r(x) = nx
\]
and the cost of producing $n$ items is given by
\[
c(x) = 2000+0.5 n. 
\]
However, from the problem we see that the number of items sold is
itself a function of $x$,
\[
n(x) =5000+\frac{1000(1.5-x)}{0.10}
\]
So profit is give by:
\begin{align*}
P(x) &= r(x) - c(x)\\
&= nx - (2000+0.5 n)\\
&=-10000x^2+25000x-12000. 
\end{align*}
We want to know the maximum value of this function when $x$ is between
0 and $1.5$. The derivative is
\[
P'(x)=\answer[given]{-20000x+25000},
\]
which is zero when $x=1.25$. Since $P''(x)=-20000<0$, there must be a
local maximum at $x=1.25$, and since this is the only critical value
it must be a global maximum as well. Alternately, we could compute
$P(0)=-12000$, $P(1.25)=3625$, and $P(1.5)=3000$ and note that
$P(1.25)$ is the maximum of these. Thus the maximum profit is
\$$3625$, attained when we set the price at \$$1.25$ and sell $7500$
items.
\begin{onlineOnly} 
   We can confirm our results by looking at the graph of $y=P(x)$:
   \[
   \graph[xmin=0,xmax=3,ymin=-4000,ymax=6000]{y=-10000x^2+25000x-12000}
   \]
\end{onlineOnly}
\end{explanation}
\end{example}








\end{document}
