\documentclass{ximera}

\input{../preamble.tex}


\title[Dig-In:]{Higher order derivatives}

\begin{document}
\begin{abstract}
 Here we look at higher order derivatives.   
\end{abstract}
\maketitle

We call the derivative of the derivative the \dfn{second
  derivative}, the derivative of the derivative of the derivative the
\dfn{third derivative}, and so on. We have special notation for
higher derivatives, check it out:
\begin{description}
\item[First derivative:] $\ddx f(x) = f'(x) = f^{(1)}(x)$.
\item[Second derivative:] $\dd[~^2]{x^2} f(x) = f''(x) = f^{(2)}(x)$.
\item[Third derivative:] $\dd[~^3]{x^3} f(x) = f'''(x) = f^{(3)}(x)$.
\end{description}



\begin{example}
Compute:
\[
\dd[~^2]{x^2} e^{x^{2}}
\]

\begin{explanation}
We need to first find $\ddx e^{x^{2}}$
\[
\ddx e^{x^{2}} =  \answer[given]{2xe^{x^{2}}}\\
\]
Then
\[
\dd[~^2]{x^2} e^{x^{2}} =  \ddx 2xe^{x^{2}}
\]
\begin{align*}
\dd[~^2]{x^2} e^{x^{2}} &=  \ddx 2xe^{x^{2}}\\ 
&= 2x \answer[given]{ 2xe^{x^{2}} }+ \answer[given]{2} 2xe^{x^{2}}
\end{align*}
\end{explanation}
\end{example}

\begin{example}
Compute:
\[
\dd[~^2]{x^2} \ln (x^{2}+2x+10)
\]

\begin{explanation}
We need to first find $\ddx  \ln (x^{2}+3x+10)$
\[
\ddx \ln (x^{2}+3x+10) =  \answer[given]{\frac{2x+3}{x^{2}+3x+10}}\\
\]
Then
\[
\dd[~^2]{x^2} \ln (x^{2}+3x+10) =  \ddx \frac{2x+3}{x^{2}+3x+10}
\]
\begin{align*}
\dd[~^2]{x^2} \ln (x^{2}+3x+10) &=  \ddx \frac{2x+3}{x^{2}+3x+10}\\ 
&=  \frac{ \answer[given]{2}(x^{2}+3x+10)+ (2x+3)\answer[given]{2x+3} }{ \answer[given]{(x^{2}+3x+10)^{2} }}
\end{align*}
\end{explanation}
\end{example}



\end{document}
