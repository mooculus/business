\documentclass{ximera}

\input{../preamble.tex}

\title[Dig-In:]{Solving Inequalities}
\begin{document}
\begin{abstract}

\end{abstract}
\maketitle

Now that we understand how a sign table is constructed, we can use these table to solve inequalities.

Let's try some examples:

\begin{example}
Solve the inequality
\[
(x-2)(x+3) < 0
\]

\begin{explanation}
We will solve this inequality by constructing a sign table for
\[ 
f(x)=(x-2)(x+3)
\]
and using this information to find the solution set.

From the previous section, we know that the sign table for $f(x)$ is given by

\begin{image}
\begin{tikzpicture}
	\begin{axis}[
            trim axis left,
            scale only axis,
            domain=-7:5,
            ymax=5,
            ymin=-5,
            axis lines=none,
            height=3cm, %% Hard coded height! 
            width=\textwidth, %% width
          ]
          %\addplot [draw=none, fill=fill1, domain=(-3:-2)] {2} \closedcycle;
          %\addplot [draw=none, fill=fill2, domain=(-2:0)] {2} \closedcycle;
          %\addplot [draw=none, fill=fill1, domain=(0:1)] {2} \closedcycle;
          %\addplot [draw=none, fill=fill2, domain=(1:3)] {2} \closedcycle;
          
          \addplot [->,textColor] plot coordinates {(-7,0) (5,0)}; %% axis{0};

                    
          \addplot [dashed, textColor] plot coordinates {(-3,0) (-3,2)};
           \addplot [dashed, textColor] plot coordinates {(2,0) (2,2)};
          
          \node at (axis cs:-3,0) [anchor=north,textColor] {\footnotesize$-3$};
          \node at (axis cs:2,0) [anchor=north,textColor] {\footnotesize$2$};

	 \node at (axis cs:-5,-2) [textColor] {\footnotesize$f(x)>0$};
          \node at (axis cs:.0,-2) [textColor] {\footnotesize$f(x)<0$};
          \node at (axis cs:4,-2) [textColor] {\footnotesize$f(x)>0$};


          %% \node at (axis cs:-2.5,-.5) [anchor=north,textColor] {\footnotesize Decreasing};
          %% \node at (axis cs:.5,-.5) [anchor=north,textColor] {\footnotesize Decreasing};
          %% \node at (axis cs:-1,-.5) [anchor=north,textColor] {\footnotesize Increasing};
          %% \node at (axis cs:2,-.5) [anchor=north,textColor] {\footnotesize Increasing};

        \end{axis}
\end{tikzpicture}
\end{image}
\end{explanation}
\end{example}

We are looking for all values of $x$ such that $f(x)<0$. Checking the sign table, we see that the solution set (in interval notation) is
\begin{selectAll}
\choice{$[-3,2]$}
\choice{$(-\infty,-3]\cup [2,\infty)$}
\choice{$(-\infty,-3)\cup (2,\infty)$}
\choice{$[-3,2)$}
\choice[correct]{$(-3,2)$}
\end{selectAll}


\begin{example}
Solve the inequality
\[
\frac{(x-4)(x+4)}{(x+2)} \geq 0
\]

\begin{explanation}
We will solve this inequality by constructing a sign table for
\[ 
f(x) = \frac{(x-4)(x+4)}{(x+2)}
\]
and using this information to find the solution set.
From the previous section, we know that the sign table for $f(x)$ is given by


\begin{image}
\begin{tikzpicture}
	\begin{axis}[
            trim axis left,
            scale only axis,
            domain=-9:7,
            ymax=5,
            ymin=-5,
            axis lines=none,
            height=3cm, %% Hard coded height! 
            width=\textwidth, %% width
          ]
          %\addplot [draw=none, fill=fill1, domain=(-3:-2)] {2} \closedcycle;
          %\addplot [draw=none, fill=fill2, domain=(-2:0)] {2} \closedcycle;
          %\addplot [draw=none, fill=fill1, domain=(0:1)] {2} \closedcycle;
          %\addplot [draw=none, fill=fill2, domain=(1:3)] {2} \closedcycle;
          
          \addplot [->,textColor] plot coordinates {(-9,0) (7,0)}; %% axis{0};

                    
          \addplot [dashed, textColor] plot coordinates {(-2,0) (-2,2)};
           \addplot [dashed, textColor] plot coordinates {(4,0) (4,2)};  
           \addplot [dashed, textColor] plot coordinates {(-5,0) (-5,2)};

          
          \node at (axis cs:-2,0) [anchor=north,textColor] {\footnotesize$-2$};
          \node at (axis cs:4,0) [anchor=north,textColor] {\footnotesize$4$};
           \node at (axis cs:-5,0) [anchor=north,textColor] {\footnotesize$-4$};

	 \node at (axis cs:-7,-2) [textColor] {\footnotesize$f(x)<0$};
         \node at (axis cs:.-3.5,-2) [textColor] {\footnotesize$f(x)>0$};
         \node at (axis cs:1,-2) [textColor] {\footnotesize$f(x)<0$};
         \node at (axis cs:6,-2) [textColor] {\footnotesize$f(x)>0$};


          %% \node at (axis cs:-2.5,-.5) [anchor=north,textColor] {\footnotesize Decreasing};
          %% \node at (axis cs:.5,-.5) [anchor=north,textColor] {\footnotesize Decreasing};
          %% \node at (axis cs:-1,-.5) [anchor=north,textColor] {\footnotesize Increasing};
          %% \node at (axis cs:2,-.5) [anchor=north,textColor] {\footnotesize Increasing};

        \end{axis}
\end{tikzpicture}
\end{image}


We are looking for all values of $x$ such that $f(x)\geq 0$. If we first limit ourselves to the points where $f(x)>0$ then we see that 

\[ f(x) > 0 
\] 
if $x$ is in the interval 
\[
(-4,-2) \text{ or } (4,\infty)
\]
Finally, as we need $f(x) \geq 0$, we also have to include the points where $f(x)$=0, 
namely $x=-4$ and $x=4$. Therefore our solution set (in interval notation) is



\begin{selectAll}
\choice{$(-\infty,-4]\cup [-2,4)$}
\choice[correct]{$[-4,-2)\cup [4,\infty)$}
\choice{$(-4,-2]\cup (4,\infty)$}
\choice{$(-\infty,-4)\cup (-2,4)$}
\choice{$(-4,-2]\cup [4,\infty)$}
\end{selectAll}


\end{explanation}
\end{example}

We'll meet sign tables again when we discuss how calculus helps us to find analytic information about a function.


\end{document}

