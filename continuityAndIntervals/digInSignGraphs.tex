\documentclass{ximera}

\input{../preamble.tex}

\title[Dig-In:]{Sign Tables}
\begin{document}
\begin{abstract}

\end{abstract}
\maketitle

Did you remember the method used to solve $(x-2)(x+3)>0$? Did it involve finding roots? One method is to collect information 
in something called a {\bf sign table} (or {\bf sign chart}, {\bf sign diagram} or {\bf sign graph}). Why this method works is a 
consequence of the intermediate value theorem:

\begin{itemize}
\item A function can change its sign only at a value where it is not continuous or at a zero. 
\end{itemize}

With this in mind, suppose that $f$ is a function which is continuous on the interval $(a,b)$ and suppose also that $f$ has no zeros in this interval:

\begin{itemize}
\item If $f(c)>0$ for some $c$ in the interval $(a,b)$ then $f(x)>0$ for all $x$ in $(a,b)$
\item If $f(c)<0$ for some $c$ in the interval $(a,b)$ then $f(x)<0$ for all $x$ in $(a,b)$
\end{itemize}

A sign table indicates the intervals where a given function is positive and where it is negative. Given $f(x)$, a sign table is produced by 
finding all values in the domain of $f(x)$ where $f(x)$ is zero or discontinuous. The values are then put on a number line. Finally, the sign of $f(x)$ between 
each of these values is found, either by sampling or from the properties if the given function.

Let's try some examples:

\begin{example}
Consider the function 
\[
f(x) = (x-2)(x+3)
\]
 Construct a sign table for $f(x)$ to indicate the intervals where $f(x)$ is positive and the intervals where $f(x)$ is negative


\begin{explanation}
First, as f(x) is a polynomial, it is continuous at all real numbers. Therefore we only need to find the zeros of $f(x)$. If 
\[
f(x) = 0
\]
then
\[ 
(x-2)(x+3) = 0
\]
So 
\[ 
(x-2)= 0 \text{ or } (x+3) = 0
\]
Which means that if $f(x)=0$ then $x=2$ or $x=-3$. With this, we indicate these values on a number line:

\begin{image}
\begin{tikzpicture}
	\begin{axis}[
            trim axis left,
            scale only axis,
            domain=-7:5,
            ymax=5,
            ymin=-5,
            axis lines=none,
            height=3cm, %% Hard coded height! 
            width=\textwidth, %% width
          ]
          %\addplot [draw=none, fill=fill1, domain=(-3:-2)] {2} \closedcycle;
          %\addplot [draw=none, fill=fill2, domain=(-2:0)] {2} \closedcycle;
          %\addplot [draw=none, fill=fill1, domain=(0:1)] {2} \closedcycle;
          %\addplot [draw=none, fill=fill2, domain=(1:3)] {2} \closedcycle;
          
          \addplot [->,textColor] plot coordinates {(-7,0) (5,0)}; %% axis{0};

                    
          \addplot [dashed, textColor] plot coordinates {(-3,0) (-3,2)};
           \addplot [dashed, textColor] plot coordinates {(2,0) (2,2)};
          
          \node at (axis cs:-3,0) [anchor=north,textColor] {\footnotesize$-3$};
          \node at (axis cs:2,0) [anchor=north,textColor] {\footnotesize$2$};

	 %\node at (axis cs:-5,-.7) [textColor] {\footnotesize$f(x)>0$};
          %\node at (axis cs:.0,-.7) [textColor] {\footnotesize$f(x)<0$};
          %\node at (axis cs:4,-.7) [textColor] {\footnotesize$f(x)>0$};


          %% \node at (axis cs:-2.5,-.5) [anchor=north,textColor] {\footnotesize Decreasing};
          %% \node at (axis cs:.5,-.5) [anchor=north,textColor] {\footnotesize Decreasing};
          %% \node at (axis cs:-1,-.5) [anchor=north,textColor] {\footnotesize Increasing};
          %% \node at (axis cs:2,-.5) [anchor=north,textColor] {\footnotesize Increasing};

        \end{axis}
\end{tikzpicture}
\end{image}


Now we can check points \textbf{between} the zeros to find
where $f(x)$ is positive and negative:
\begin{align*}
  f(-4)&=\answer[given]{8},\\
  f(0)&=\answer[given]{-6},\\
  f(3)&=\answer[given]{6}.
\end{align*}
From this we can make a sign table:

\begin{image}
\begin{tikzpicture}
	\begin{axis}[
            trim axis left,
            scale only axis,
            domain=-7:5,
            ymax=5,
            ymin=-5,
            axis lines=none,
            height=3cm, %% Hard coded height! 
            width=\textwidth, %% width
          ]
          %\addplot [draw=none, fill=fill1, domain=(-3:-2)] {2} \closedcycle;
          %\addplot [draw=none, fill=fill2, domain=(-2:0)] {2} \closedcycle;
          %\addplot [draw=none, fill=fill1, domain=(0:1)] {2} \closedcycle;
          %\addplot [draw=none, fill=fill2, domain=(1:3)] {2} \closedcycle;
          
          \addplot [->,textColor] plot coordinates {(-7,0) (5,0)}; %% axis{0};

                    
          \addplot [dashed, textColor] plot coordinates {(-3,0) (-3,2)};
           \addplot [dashed, textColor] plot coordinates {(2,0) (2,2)};
          
          \node at (axis cs:-3,0) [anchor=north,textColor] {\footnotesize$-3$};
          \node at (axis cs:2,0) [anchor=north,textColor] {\footnotesize$2$};

	 \node at (axis cs:-5,-2) [textColor] {\footnotesize$f(x)>0$};
          \node at (axis cs:.0,-2) [textColor] {\footnotesize$f(x)<0$};
          \node at (axis cs:4,-2) [textColor] {\footnotesize$f(x)>0$};


          %% \node at (axis cs:-2.5,-.5) [anchor=north,textColor] {\footnotesize Decreasing};
          %% \node at (axis cs:.5,-.5) [anchor=north,textColor] {\footnotesize Decreasing};
          %% \node at (axis cs:-1,-.5) [anchor=north,textColor] {\footnotesize Increasing};
          %% \node at (axis cs:2,-.5) [anchor=north,textColor] {\footnotesize Increasing};

        \end{axis}
\end{tikzpicture}
\end{image}


\end{explanation}
\end{example}
\begin{example}
Consider the function 
\[
f(x) = \frac{(x-4)(x+4)}{(x+2)}
\]
 Construct a sign table for $f(x)$ to indicate the intervals where $f(x)$ is positive and the intervals where $f(x)$ is negative


\begin{explanation}
First, as f(x) is a rational function, it is continuous at all values in its domain. In particular, the only values where $f(x)$ is not continuous is
where $f(x)$ is undefined. This will happen when its denominator is zero: 

\begin{align*}
  (x+2) &= 0 \\
 x &=\answer[given]{-2}
\end{align*}

So we will have $x=2$ as one of our labels on our sign table. Next, we need to find where $f(x)=0$. Here:

\[
  (x-4)(x+4) = 0 
\]
So $f(x)=0$ when $x=4$ or $x=-4$

Now we can label our sign table

\begin{image}
\begin{tikzpicture}
	\begin{axis}[
            trim axis left,
            scale only axis,
            domain=-7:7,
            ymax=5,
            ymin=-5,
            axis lines=none,
            height=3cm, %% Hard coded height! 
            width=\textwidth, %% width
          ]
          %\addplot [draw=none, fill=fill1, domain=(-3:-2)] {2} \closedcycle;
          %\addplot [draw=none, fill=fill2, domain=(-2:0)] {2} \closedcycle;
          %\addplot [draw=none, fill=fill1, domain=(0:1)] {2} \closedcycle;
          %\addplot [draw=none, fill=fill2, domain=(1:3)] {2} \closedcycle;
          
          \addplot [->,textColor] plot coordinates {(-7,0) (7,0)}; %% axis{0};

                    
          \addplot [dashed, textColor] plot coordinates {(-2,0) (-2,2)};
           \addplot [dashed, textColor] plot coordinates {(4,0) (4,2)};  
           \addplot [dashed, textColor] plot coordinates {(-5,0) (-5,2)};

          
          \node at (axis cs:-2,0) [anchor=north,textColor] {\footnotesize$-2$};
          \node at (axis cs:4,0) [anchor=north,textColor] {\footnotesize$4$};
           \node at (axis cs:-5,0) [anchor=north,textColor] {\footnotesize$-4$};

	 %\node at (axis cs:-5,-.7) [textColor] {\footnotesize$f(x)>0$};
          %\node at (axis cs:.0,-.7) [textColor] {\footnotesize$f(x)<0$};
          %\node at (axis cs:4,-.7) [textColor] {\footnotesize$f(x)>0$};


          %% \node at (axis cs:-2.5,-.5) [anchor=north,textColor] {\footnotesize Decreasing};
          %% \node at (axis cs:.5,-.5) [anchor=north,textColor] {\footnotesize Decreasing};
          %% \node at (axis cs:-1,-.5) [anchor=north,textColor] {\footnotesize Increasing};
          %% \node at (axis cs:2,-.5) [anchor=north,textColor] {\footnotesize Increasing};

        \end{axis}
\end{tikzpicture}
\end{image}


Now we can check points \textbf{between} the zeros and where $f(x)$ is not continuous to find
where $f(x)$ is positive and negative:
\begin{align*}
  f(-5)&=\answer[given]{-3},\\
  f(-3)&=\answer[given]{7},\\
   f(0)&=\answer[given]{-8},\\
  f(5)&=\answer[given]{\frac{9}{7}}.
\end{align*}
From this we can make a sign table:

\begin{image}
\begin{tikzpicture}
	\begin{axis}[
            trim axis left,
            scale only axis,
            domain=-9:7,
            ymax=5,
            ymin=-5,
            axis lines=none,
            height=3cm, %% Hard coded height! 
            width=\textwidth, %% width
          ]
          %\addplot [draw=none, fill=fill1, domain=(-3:-2)] {2} \closedcycle;
          %\addplot [draw=none, fill=fill2, domain=(-2:0)] {2} \closedcycle;
          %\addplot [draw=none, fill=fill1, domain=(0:1)] {2} \closedcycle;
          %\addplot [draw=none, fill=fill2, domain=(1:3)] {2} \closedcycle;
          
          \addplot [->,textColor] plot coordinates {(-9,0) (7,0)}; %% axis{0};

                    
          \addplot [dashed, textColor] plot coordinates {(-2,0) (-2,2)};
           \addplot [dashed, textColor] plot coordinates {(4,0) (4,2)};  
           \addplot [dashed, textColor] plot coordinates {(-5,0) (-5,2)};

          
          \node at (axis cs:-2,0) [anchor=north,textColor] {\footnotesize$-2$};
          \node at (axis cs:4,0) [anchor=north,textColor] {\footnotesize$4$};
           \node at (axis cs:-5,0) [anchor=north,textColor] {\footnotesize$-4$};

	 \node at (axis cs:-7,-2) [textColor] {\footnotesize$f(x)<0$};
         \node at (axis cs:.-3.5,-2) [textColor] {\footnotesize$f(x)>0$};
         \node at (axis cs:1,-2) [textColor] {\footnotesize$f(x)<0$};
         \node at (axis cs:6,-2) [textColor] {\footnotesize$f(x)>0$};


          %% \node at (axis cs:-2.5,-.5) [anchor=north,textColor] {\footnotesize Decreasing};
          %% \node at (axis cs:.5,-.5) [anchor=north,textColor] {\footnotesize Decreasing};
          %% \node at (axis cs:-1,-.5) [anchor=north,textColor] {\footnotesize Increasing};
          %% \node at (axis cs:2,-.5) [anchor=north,textColor] {\footnotesize Increasing};

        \end{axis}
\end{tikzpicture}
\end{image}


\end{explanation}
\end{example}


\end{document}

