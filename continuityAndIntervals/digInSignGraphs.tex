\documentclass{ximera}

%\usepackage{todonotes}

\newcommand{\todo}{}

%\usepackage{esint} % for \oiint
\graphicspath{
{./}
{functionsOfSeveralVariables/}
{normalVectors/}
{lagrangeMultipliers/}
{vectorFields/}
{greensTheorem/}
{shapeOfThingsToCome/}
}


\usepackage{tkz-euclide}

\tikzset{>=stealth} %% cool arrow head
\tikzset{shorten <>/.style={ shorten >=#1, shorten <=#1 } } %% allows shorter vectors

\usetikzlibrary{backgrounds} %% for boxes around graphs
\usetikzlibrary{shapes,positioning}  %% Clouds and stars
\usetikzlibrary{matrix} %% for matrix
\usetikzlibrary{patterns} 
\usepgfplotslibrary{polar} %% for polar plots

\usepackage[makeroom]{cancel} %% for strike outs
%\usepackage{mathtools} %% for pretty underbrace % Breaks Ximera
\usepackage{multicol}
\usepackage{pgffor} %% required for integral for loops


%% http://tex.stackexchange.com/questions/66490/drawing-a-tikz-arc-specifying-the-center
%% Draws beach ball
\tikzset{pics/carc/.style args={#1:#2:#3}{code={\draw[pic actions] (#1:#3) arc(#1:#2:#3);}}}



\usepackage{array}
\setlength{\extrarowheight}{+.1cm}   
\newdimen\digitwidth
\settowidth\digitwidth{9}
\def\divrule#1#2{
\noalign{\moveright#1\digitwidth
\vbox{\hrule width#2\digitwidth}}}





\newcommand{\RR}{\mathbb R}
\newcommand{\R}{\mathbb R}
\newcommand{\N}{\mathbb N}
\newcommand{\Z}{\mathbb Z}

%\newcommand{\sage}{\textsf{SageMath}}


%\renewcommand{\d}{\,d\!}
\renewcommand{\d}{\mathop{}\!d}
\newcommand{\dd}[2][]{\frac{\d #1}{\d #2}}
\newcommand{\pp}[2][]{\frac{\partial #1}{\partial #2}}
\renewcommand{\l}{\ell}
\newcommand{\ddx}{\frac{d}{\d x}}

\newcommand{\zeroOverZero}{\ensuremath{\boldsymbol{\tfrac{0}{0}}}}
\newcommand{\inftyOverInfty}{\ensuremath{\boldsymbol{\tfrac{\infty}{\infty}}}}
\newcommand{\zeroOverInfty}{\ensuremath{\boldsymbol{\tfrac{0}{\infty}}}}
\newcommand{\zeroTimesInfty}{\ensuremath{\small\boldsymbol{0\cdot \infty}}}
\newcommand{\inftyMinusInfty}{\ensuremath{\small\boldsymbol{\infty - \infty}}}
\newcommand{\oneToInfty}{\ensuremath{\boldsymbol{1^\infty}}}
\newcommand{\zeroToZero}{\ensuremath{\boldsymbol{0^0}}}
\newcommand{\inftyToZero}{\ensuremath{\boldsymbol{\infty^0}}}



\newcommand{\numOverZero}{\ensuremath{\boldsymbol{\tfrac{\#}{0}}}}
\newcommand{\dfn}{\textbf}
%\newcommand{\unit}{\,\mathrm}
\newcommand{\unit}{\mathop{}\!\mathrm}
\newcommand{\eval}[1]{\bigg[ #1 \bigg]}
\newcommand{\seq}[1]{\left( #1 \right)}
\renewcommand{\epsilon}{\varepsilon}
\renewcommand{\phi}{\varphi}


\renewcommand{\iff}{\Leftrightarrow}

\DeclareMathOperator{\arccot}{arccot}
\DeclareMathOperator{\arcsec}{arcsec}
\DeclareMathOperator{\arccsc}{arccsc}
\DeclareMathOperator{\si}{Si}
\DeclareMathOperator{\proj}{\vec{proj}}
\DeclareMathOperator{\scal}{scal}
\DeclareMathOperator{\sign}{sign}


%% \newcommand{\tightoverset}[2]{% for arrow vec
%%   \mathop{#2}\limits^{\vbox to -.5ex{\kern-0.75ex\hbox{$#1$}\vss}}}
\newcommand{\arrowvec}{\overrightarrow}
%\renewcommand{\vec}[1]{\arrowvec{\mathbf{#1}}}
\renewcommand{\vec}{\mathbf}
\newcommand{\veci}{{\boldsymbol{\hat{\imath}}}}
\newcommand{\vecj}{{\boldsymbol{\hat{\jmath}}}}
\newcommand{\veck}{{\boldsymbol{\hat{k}}}}
\newcommand{\vecl}{\boldsymbol{\l}}
\newcommand{\uvec}[1]{\mathbf{\hat{#1}}}
\newcommand{\utan}{\mathbf{\hat{t}}}
\newcommand{\unormal}{\mathbf{\hat{n}}}
\newcommand{\ubinormal}{\mathbf{\hat{b}}}

\newcommand{\dotp}{\bullet}
\newcommand{\cross}{\boldsymbol\times}
\newcommand{\grad}{\boldsymbol\nabla}
\newcommand{\divergence}{\grad\dotp}
\newcommand{\curl}{\grad\cross}
%\DeclareMathOperator{\divergence}{divergence}
%\DeclareMathOperator{\curl}[1]{\grad\cross #1}
\newcommand{\lto}{\mathop{\longrightarrow\,}\limits}

\renewcommand{\bar}{\overline}

\colorlet{textColor}{black} 
\colorlet{background}{white}
\colorlet{penColor}{blue!50!black} % Color of a curve in a plot
\colorlet{penColor2}{red!50!black}% Color of a curve in a plot
\colorlet{penColor3}{red!50!blue} % Color of a curve in a plot
\colorlet{penColor4}{green!50!black} % Color of a curve in a plot
\colorlet{penColor5}{orange!80!black} % Color of a curve in a plot
\colorlet{penColor6}{yellow!70!black} % Color of a curve in a plot
\colorlet{fill1}{penColor!20} % Color of fill in a plot
\colorlet{fill2}{penColor2!20} % Color of fill in a plot
\colorlet{fillp}{fill1} % Color of positive area
\colorlet{filln}{penColor2!20} % Color of negative area
\colorlet{fill3}{penColor3!20} % Fill
\colorlet{fill4}{penColor4!20} % Fill
\colorlet{fill5}{penColor5!20} % Fill
\colorlet{gridColor}{gray!50} % Color of grid in a plot

\newcommand{\surfaceColor}{violet}
\newcommand{\surfaceColorTwo}{redyellow}
\newcommand{\sliceColor}{greenyellow}




\pgfmathdeclarefunction{gauss}{2}{% gives gaussian
  \pgfmathparse{1/(#2*sqrt(2*pi))*exp(-((x-#1)^2)/(2*#2^2))}%
}


%%%%%%%%%%%%%
%% Vectors
%%%%%%%%%%%%%

%% Simple horiz vectors
\renewcommand{\vector}[1]{\left\langle #1\right\rangle}


%% %% Complex Horiz Vectors with angle brackets
%% \makeatletter
%% \renewcommand{\vector}[2][ , ]{\left\langle%
%%   \def\nextitem{\def\nextitem{#1}}%
%%   \@for \el:=#2\do{\nextitem\el}\right\rangle%
%% }
%% \makeatother

%% %% Vertical Vectors
%% \def\vector#1{\begin{bmatrix}\vecListA#1,,\end{bmatrix}}
%% \def\vecListA#1,{\if,#1,\else #1\cr \expandafter \vecListA \fi}

%%%%%%%%%%%%%
%% End of vectors
%%%%%%%%%%%%%

%\newcommand{\fullwidth}{}
%\newcommand{\normalwidth}{}



%% makes a snazzy t-chart for evaluating functions
%\newenvironment{tchart}{\rowcolors{2}{}{background!90!textColor}\array}{\endarray}

%%This is to help with formatting on future title pages.
\newenvironment{sectionOutcomes}{}{} 



%% Flowchart stuff
%\tikzstyle{startstop} = [rectangle, rounded corners, minimum width=3cm, minimum height=1cm,text centered, draw=black]
%\tikzstyle{question} = [rectangle, minimum width=3cm, minimum height=1cm, text centered, draw=black]
%\tikzstyle{decision} = [trapezium, trapezium left angle=70, trapezium right angle=110, minimum width=3cm, minimum height=1cm, text centered, draw=black]
%\tikzstyle{question} = [rectangle, rounded corners, minimum width=3cm, minimum height=1cm,text centered, draw=black]
%\tikzstyle{process} = [rectangle, minimum width=3cm, minimum height=1cm, text centered, draw=black]
%\tikzstyle{decision} = [trapezium, trapezium left angle=70, trapezium right angle=110, minimum width=3cm, minimum height=1cm, text centered, draw=black]


\title[Dig-In:]{Sign Tables}
\begin{document}
\begin{abstract}

\end{abstract}
\maketitle

Did you remember the method used to solve $(x-2)(x+3)>0$? Did it involve finding roots? One method is to collect information 
in something called a {\bf sign table} (or {\bf sign chart}, {\bf sign diagram} or {\bf sign graph}). Why this method works is a 
consequence of the intermediate value theorem:

\begin{itemize}
\item A function can change its sign only at a value where it is not continuous or at a zero. 
\end{itemize}

With this in mind, suppose that $f$ is a function which is continuous on the interval $(a,b)$ and suppose also that $f$ has no zeros in this interval:

\begin{itemize}
\item If $f(c)>0$ for some $c$ in the interval $(a,b)$ then $f(x)>0$ for all $x$ in $(a,b)$
\item If $f(c)<0$ for some $c$ in the interval $(a,b)$ then $f(x)<0$ for all $x$ in $(a,b)$
\end{itemize}

A sign table indicates the intervals where a given function is positive and where it is negative. Given $f(x)$, a sign table is produced by 
finding all values in the domain of $f(x)$ where $f(x)$ is zero or discontinuous. The values are then put on a number line. Finally, the sign of $f(x)$ between 
each of these values is found, either by sampling or from the properties if the given function.

Let's try some examples:

\begin{example}
Consider the function 
\[
f(x) = (x-2)(x+3)
\]
 Construct a sign table for $f(x)$ to indicate the intervals where $f(x)$ is positive and the intervals where $f(x)$ is negative


\begin{explanation}
First, as f(x) is a polynomial, it is continuous at all real numbers. Therefore we only need to find the zeros of $f(x)$. If 
\[
f(x) = 0
\]
then
\[ 
(x-2)(x+3) = 0
\]
So 
\[ 
(x-2)= 0 \text{ or } (x+3) = 0
\]
Which means that if $f(x)=0$ then $x=2$ or $x=-3$. With this, we indicate these values on a number line:

\begin{image}
\begin{tikzpicture}
	\begin{axis}[
            trim axis left,
            scale only axis,
            domain=-7:5,
            ymax=5,
            ymin=-5,
            axis lines=none,
            height=3cm, %% Hard coded height! 
            width=\textwidth, %% width
          ]
          %\addplot [draw=none, fill=fill1, domain=(-3:-2)] {2} \closedcycle;
          %\addplot [draw=none, fill=fill2, domain=(-2:0)] {2} \closedcycle;
          %\addplot [draw=none, fill=fill1, domain=(0:1)] {2} \closedcycle;
          %\addplot [draw=none, fill=fill2, domain=(1:3)] {2} \closedcycle;
          
          \addplot [->,textColor] plot coordinates {(-7,0) (5,0)}; %% axis{0};

                    
          \addplot [dashed, textColor] plot coordinates {(-3,0) (-3,2)};
           \addplot [dashed, textColor] plot coordinates {(2,0) (2,2)};
          
          \node at (axis cs:-3,0) [anchor=north,textColor] {\footnotesize$-3$};
          \node at (axis cs:2,0) [anchor=north,textColor] {\footnotesize$2$};

	 %\node at (axis cs:-5,-.7) [textColor] {\footnotesize$f(x)>0$};
          %\node at (axis cs:.0,-.7) [textColor] {\footnotesize$f(x)<0$};
          %\node at (axis cs:4,-.7) [textColor] {\footnotesize$f(x)>0$};


          %% \node at (axis cs:-2.5,-.5) [anchor=north,textColor] {\footnotesize Decreasing};
          %% \node at (axis cs:.5,-.5) [anchor=north,textColor] {\footnotesize Decreasing};
          %% \node at (axis cs:-1,-.5) [anchor=north,textColor] {\footnotesize Increasing};
          %% \node at (axis cs:2,-.5) [anchor=north,textColor] {\footnotesize Increasing};

        \end{axis}
\end{tikzpicture}
\end{image}


Now we can check points \textbf{between} the zeros to find
where $f(x)$ is positive and negative:
\begin{align*}
  f(-4)&=\answer[given]{6},\\
  f(0)&=\answer[given]{-6},\\
  f(3)&=\answer[given]{6}.
\end{align*}
From this we can make a sign table:

\begin{image}
\begin{tikzpicture}
	\begin{axis}[
            trim axis left,
            scale only axis,
            domain=-7:5,
            ymax=5,
            ymin=-5,
            axis lines=none,
            height=3cm, %% Hard coded height! 
            width=\textwidth, %% width
          ]
          %\addplot [draw=none, fill=fill1, domain=(-3:-2)] {2} \closedcycle;
          %\addplot [draw=none, fill=fill2, domain=(-2:0)] {2} \closedcycle;
          %\addplot [draw=none, fill=fill1, domain=(0:1)] {2} \closedcycle;
          %\addplot [draw=none, fill=fill2, domain=(1:3)] {2} \closedcycle;
          
          \addplot [->,textColor] plot coordinates {(-7,0) (5,0)}; %% axis{0};

                    
          \addplot [dashed, textColor] plot coordinates {(-3,0) (-3,2)};
           \addplot [dashed, textColor] plot coordinates {(2,0) (2,2)};
          
          \node at (axis cs:-3,0) [anchor=north,textColor] {\footnotesize$-3$};
          \node at (axis cs:2,0) [anchor=north,textColor] {\footnotesize$2$};

	 \node at (axis cs:-5,-2) [textColor] {\footnotesize$f(x)>0$};
          \node at (axis cs:.0,-2) [textColor] {\footnotesize$f(x)<0$};
          \node at (axis cs:4,-2) [textColor] {\footnotesize$f(x)>0$};


          %% \node at (axis cs:-2.5,-.5) [anchor=north,textColor] {\footnotesize Decreasing};
          %% \node at (axis cs:.5,-.5) [anchor=north,textColor] {\footnotesize Decreasing};
          %% \node at (axis cs:-1,-.5) [anchor=north,textColor] {\footnotesize Increasing};
          %% \node at (axis cs:2,-.5) [anchor=north,textColor] {\footnotesize Increasing};

        \end{axis}
\end{tikzpicture}
\end{image}


\end{explanation}
\end{example}
\begin{example}
Consider the function 
\[
f(x) = \frac{(x-4)(x+4)}{(x+2)}
\]
 Construct a sign table for $f(x)$ to indicate the intervals where $f(x)$ is positive and the intervals where $f(x)$ is negative


\begin{explanation}
First, as f(x) is a rational function, it is continuous at all values in its domain. In particular, the only values where $f(x)$ is not continuous is
where $f(x)$ is undefined. This will happen when its denominator is zero: 

\begin{align*}
  (x+2) &= 0 \\
 x &=\answer[given]{-2}
\end{align*}

So we will have $x=2$ as one of our labels on our sign table. Next, we need to find where $f(x)=0$. Here:

\[
  (x-4)(x+4) = 0 
\]
So $f(x)=0$ when $x=4$ or $x=-4$

Now we can label our sign table

\begin{image}
\begin{tikzpicture}
	\begin{axis}[
            trim axis left,
            scale only axis,
            domain=-7:7,
            ymax=5,
            ymin=-5,
            axis lines=none,
            height=3cm, %% Hard coded height! 
            width=\textwidth, %% width
          ]
          %\addplot [draw=none, fill=fill1, domain=(-3:-2)] {2} \closedcycle;
          %\addplot [draw=none, fill=fill2, domain=(-2:0)] {2} \closedcycle;
          %\addplot [draw=none, fill=fill1, domain=(0:1)] {2} \closedcycle;
          %\addplot [draw=none, fill=fill2, domain=(1:3)] {2} \closedcycle;
          
          \addplot [->,textColor] plot coordinates {(-7,0) (7,0)}; %% axis{0};

                    
          \addplot [dashed, textColor] plot coordinates {(-2,0) (-2,2)};
           \addplot [dashed, textColor] plot coordinates {(4,0) (4,2)};  
           \addplot [dashed, textColor] plot coordinates {(-5,0) (-5,2)};

          
          \node at (axis cs:-2,0) [anchor=north,textColor] {\footnotesize$-2$};
          \node at (axis cs:4,0) [anchor=north,textColor] {\footnotesize$4$};
           \node at (axis cs:-5,0) [anchor=north,textColor] {\footnotesize$-4$};

	 %\node at (axis cs:-5,-.7) [textColor] {\footnotesize$f(x)>0$};
          %\node at (axis cs:.0,-.7) [textColor] {\footnotesize$f(x)<0$};
          %\node at (axis cs:4,-.7) [textColor] {\footnotesize$f(x)>0$};


          %% \node at (axis cs:-2.5,-.5) [anchor=north,textColor] {\footnotesize Decreasing};
          %% \node at (axis cs:.5,-.5) [anchor=north,textColor] {\footnotesize Decreasing};
          %% \node at (axis cs:-1,-.5) [anchor=north,textColor] {\footnotesize Increasing};
          %% \node at (axis cs:2,-.5) [anchor=north,textColor] {\footnotesize Increasing};

        \end{axis}
\end{tikzpicture}
\end{image}


Now we can check points \textbf{between} the zeros and where $f(x)$ is not continuous to find
where $f(x)$ is positive and negative:
\begin{align*}
  f(-5)&=\answer[given]{-3},\\
  f(-3)&=\answer[given]{7},\\
   f(0)&=\answer[given]{-8},\\
  f(5)&=\answer[given]{\frac{9}{7}}.
\end{align*}
From this we can make a sign table:

\begin{image}
\begin{tikzpicture}
	\begin{axis}[
            trim axis left,
            scale only axis,
            domain=-9:7,
            ymax=5,
            ymin=-5,
            axis lines=none,
            height=3cm, %% Hard coded height! 
            width=\textwidth, %% width
          ]
          %\addplot [draw=none, fill=fill1, domain=(-3:-2)] {2} \closedcycle;
          %\addplot [draw=none, fill=fill2, domain=(-2:0)] {2} \closedcycle;
          %\addplot [draw=none, fill=fill1, domain=(0:1)] {2} \closedcycle;
          %\addplot [draw=none, fill=fill2, domain=(1:3)] {2} \closedcycle;
          
          \addplot [->,textColor] plot coordinates {(-9,0) (7,0)}; %% axis{0};

                    
          \addplot [dashed, textColor] plot coordinates {(-2,0) (-2,2)};
           \addplot [dashed, textColor] plot coordinates {(4,0) (4,2)};  
           \addplot [dashed, textColor] plot coordinates {(-5,0) (-5,2)};

          
          \node at (axis cs:-2,0) [anchor=north,textColor] {\footnotesize$-2$};
          \node at (axis cs:4,0) [anchor=north,textColor] {\footnotesize$4$};
           \node at (axis cs:-5,0) [anchor=north,textColor] {\footnotesize$-4$};

	 \node at (axis cs:-7,-2) [textColor] {\footnotesize$f(x)<0$};
         \node at (axis cs:.-3.5,-2) [textColor] {\footnotesize$f(x)>0$};
         \node at (axis cs:1,-2) [textColor] {\footnotesize$f(x)<0$};
         \node at (axis cs:6,-2) [textColor] {\footnotesize$f(x)>0$};


          %% \node at (axis cs:-2.5,-.5) [anchor=north,textColor] {\footnotesize Decreasing};
          %% \node at (axis cs:.5,-.5) [anchor=north,textColor] {\footnotesize Decreasing};
          %% \node at (axis cs:-1,-.5) [anchor=north,textColor] {\footnotesize Increasing};
          %% \node at (axis cs:2,-.5) [anchor=north,textColor] {\footnotesize Increasing};

        \end{axis}
\end{tikzpicture}
\end{image}


\end{explanation}
\end{example}


\end{document}

